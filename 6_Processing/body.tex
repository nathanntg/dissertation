\chapter{Interface and software for near-real-time processing and feedback}

\label{chapter:processing}
\thispagestyle{myheadings}

% path to figures for this chapter
% must have a trailing slash
\graphicspath{{6_Processing/Figures/}}

\section{Introduction}

% TODO: introduction

\section{Methods}

% TODO: introduce methods section

\subsection{Chronic fiber interface}

% TODO: revise paragraph
In order to deploy the fiber bundles in awake behaving experimental 
paradigms, we envision an implementation of the bundle interface where
 the animal is housed in an enclosure below a traditional fluorescence
 microscope with the ensheathed bundle running from the microscope to 
the implant point (see figure x). Depending on the duration of the 
experiment, the fiber can be fixed below the microscope objective or 
placed on a low-friction passive commutator (e.g., using precision 
ball bearings) that allows the fiber bundle to rotate as the animal 
moves. Given the flexibility of the fiber bundle, it may be necessary 
to add a second material that helps translate the rotational torque to
 the commutator. During post-processing of the acquired recordings, 
the rotation of the fiber bundle can be corrected through 
registration.

\subsection{Acquisition hardware}

% TODO: describe acquisition hardware

\subsection{Acquisition software}

% TODO: describe two software acquisition approaches

\subsection{Near-real-time software}

To manage and trigger acquisition, and to do near-real-time signal
processing, we created custom image acquisition software that runs
on the macOS operating system (\fref{fig:software}). Leveraging the 
native AVFoundation
framework, the software is capable of communicating with USB video 
acquisition hardware, including both analog-to-digital frame 
grabbers and cameras. The software can captured video or video 
synchronized with a multi-channel analog data stream (including 
either audio or other analog data). Video and analog data are 
written to disk in MPEG-4 container files with video encoded at 
full resolution either uncompressed or using the H.264 or lossless 
MJPEG Open DML codecs and audio encoded using either the AAC codec 
or the linear PCM format with a sampling rate of up to 
96~\si{\kilo\hertz}. Simultaneously, the software writes metadata 
including frame timing, region of interest intensity and feedback 
parameters to a text table (CSV format).

\begin{figure}
\includegraphics[width=\textwidth]{fig2-software.png}
\caption[Software for capture near-real-time analysis]{\textbf{User interface of acquisition
control software capable of near-real-time analysis.} The software 
controls capture related peripherals, including USB camera sensors,
light sources and microcontroller breakout boxes, and manages 
synchronized acquisition of video and analog signals. During 
acquisition, the software can process video frames, extracting 
multiple annotated regions of interest, and can process analog 
signals, such as applying audio template matching. Analysis 
can trigger external feedback, acting as a basic brain machine 
interface.}
\label{fig:software}
\end{figure}

In addition, the software communicates with a microcontroller 
(Arduino Uno Rev3 or 2560 Mega) via a USB-to-serial connection. The 
microcontroller allows the software to control additional 
peripherals, including light source brightness, external acquisition 
cameras and the commutator.

The software is able to perform near real-time analysis on both the 
video frames and the analog input channels---these channels enable
low-latency, video-aligned acquisition of audio, external TTL 
pulses, electrophysiology or data regarding behavioral context. 
For example, in our experiments recording fluorescence in songbirds, 
we use the analog channels to acquire audio and trigger acquisition 
based off song detection (by processing audio through a short-time
Fourier transform to calculate the ratio of power in frequency 
bins associated with song [1--10~kHz] to the power of frequency bins outside 
the song range). The microcontroller is used to activate 
the excitation light source, restricting illumination to periods of 
singing and, as a result, restricting photobleaching. 

During monitoring and acquisition, the software can perform low-latency
computational analysis of both the video frames and analog input. In 
the case of the video, the user interface provides the ability to 
draw regions of interest (ROIs) onto the field of view and processing 
rules in the form of symbolic equations that can be used to trigger 
external feedback (e.g., sound or reward). At the start of
acquisition, the software converts the regions of interest into indices 
corresponding with the ROI masks and parses the symbolic equations 
into an abstract syntax tree (AST) for optimization and evaluation. 
Each frame is read in YCbCr color 
space, providing direct access to the luma component (Y). Using vector
optimized instructions when available, the software accumulates average 
pixel intensity for each region of interest and then evaluates the 
syntax tree and outputs the appropriate feedback value. \Fref{fig:software} 
shows the software configured to compare $\Delta$F values for two 
regions of interest corresponding with two fibers.

In addition, the software can perform low-latency computational 
analysis of the analog inputs. Given the broader range of analog 
applications and analysis, the software does not have a graphical 
user interface for defining analysis, but the code provides clear 
points for extension. We have extended and open sourced versions 
of the software that use the analog signals to acquire synchronized 
audio. Acquired audio is converted to a spectral representation 
through a two-taper short-time Fourier transform (STFT). These 
analysis steps leverage the vDSP Accelerate framework, which 
relies on vectorized instructions whenever possible and achieving 
low latency processing. We use the spectral representation to 
trigger acquisition (during singing). In addition, the spectral
representation can serve as an input to a neural network trained 
to precisely identify salient audio (i.e., a syllable or other 
vocalization) \cite{Pearre:2017cs}; this enables providing 
aversive white noise feedback during syllable presentation.

To evaluate performance of the software, we blink an LED in the 
field of view of the camera, while recording the output trigger 
controlled by the software to detect when the LED intensity is 
detectable. The LED activation triggers activation on an 
oscilloscope, which is to averaging mode (128-trigger average) 
in order to collect latency data. In addition, external logging
flags in the software are used to evaluate processing time for 
specific steps in the analysis pipeline (such as ROI extraction).
Audio processing is evaluated as described in \cite{Pearre:2017cs},
using recordings of songs with a ground truth annotation on the 
second channel. The ground truth signal triggers oscilloscope 
acquisition, again averaging over 128 trials to build a distribution 
of timing performance.

\subsection{Synchronized acquisition}

% TODO: show synchronized acquisition circuit

\subsection{Analysis pipeline}

% TODO: describe node based analysis pipeline

\section{Results}

% TODO: write some results
% commutator
% software latency

\subsection{Analysis performance}

To evaluate the near-real-time analysis and feedback, we recorded 
signals that provide a reference signal that represents the 
ground truth. 

% TODO: ideally remeasure and generate new figure
Region of interest (ROI) video analysis was evaluated by measuring the
time from LED onset to software feedback. The average time to trigger 
a feedback stimulus is 23.9 $\pm$ 7.9~ms (95\% confidence interval), 
including frame exposure, digitization, transmission, data acquisition 
and software processing. This latency reflects the intrinsic 
limitations of the 33~ms exposure time (which accounts for 16.5~ms of 
the delay) as well as the time spent capturing and decoding the video, 
and communicating an output TTL via the microcontroller. 
\Fref{fig:video-latency} depicts video controller latency.

\begin{figure}
\includegraphics[width=\textwidth]{fig3-video-latency.eps}
\caption[Latency for video processing]{\textbf{Video analysis 
produces feedback in 24~ms, including exposure time.} 
% TODO: finish caption

 C. Example of feedback contingent on ROI tracking. Black: voltage driving an LED light flash that is recorded in the field of the CMOS; blue: the cumulative probability density function (CDF) of a brief TTL pulse triggered by the software in response to the LED flash processed through the entire acquisition system. Event detection was based on ROI analysis on a Mac Mini computer. Latency of the full loop from camera to Arduino based TTL output is approximately 23.9 ms ± 7.9 ms (95\% confidence interval), with the jitter comparable to the frame rate of the camera. In this test, the LED was not synchronized to the onset of the frame, as would be the case for spontaneous video recording of neural activity. This represents the experimentally relevant performance of the system, intrinsically limited by the 33 ms frame rate of the camera. D. Of the total latency, image processing to extract fluorescence from ten cell-sized regions of interest contributes an average of only 0.17 ms; much of the ROI feedback latency is a reflection of the frame rate and acquisition time.

}
\label{fig:video-latency}
\end{figure}


Figure 2C depicts time between LED onset and feedback in our software 
test bed. The average time to trigger a feedback stimulus is 23.9 
$\pm$ 7.9~ms (95\% confidence interval), including frame exposure, 
digitization, transmission, data acquisition and software processing. 
This latency reflects the intrinsic limitations of the 30 FPS (33~ms) 
frame rate as well as time spent capturing and decoding video, and 
communicating an output TTL with the Arduino controller through the 
serial port. Data acquisition, region of interest processing, and 
feedback triggering all occur within an average of 1.7~ms after frame 
capture (95\% confidence: <11~ms). This variability stems primarily 
from video encoding and storage, where the codec and write buffering 
necessitate increased processing time for a subset of frames. An 
additional  source of variability in processing time, which has a 
smaller impact and is shown Figure 2D, relates to the number and 
complexity of the defined regions of interest. As the number and size 
of the regions of interest increases, the processing time increases. 
In tests involving ten cell-sized regions of interest, the average 
processing time was 0.17~ms with a range of  0.11--0.22~ms (95\% 
confidence interval).

\begin{figure}
\includegraphics[width=\textwidth]{fig4-audio-latency.eps}
\caption[Latency for audio processing]{\textbf{Audio analysis 
produces feedback output in under 5~ms.} 
% TODO: finish caption
}
\label{fig:audio-latency}
\end{figure}

\section{Discussion}

% TODO: write a discussion

 In this paradigm, the camera can be used as a brain machine interface
 (BMI) in which songbirds control sounds directly through the measured
 calcium signals in the brain (Clancy et al. 2014). Other applications
 of the software could include closed-loop stimulation experiments 
that seek to electrically or optically disrupt patterns of activity in
 real-time.

In our tests, the latency of triggering off fluorescence activity was 
limited by the framerate of the microscope, about 23.9 $\pm$ 7.9~ms 
Figure 2C. For many experiments, this response latency is acceptable. 
In the songbird, for example, the sensory-motor latency from pre-motor
 neuron activity in HVC to auditory sensory consequences processed in 
the basal ganglia is a minimum of 32--50~ms (Andalman \& Fee 2009). 
Based on these estimates and published accounts of conditional 
feedback experiments in songbirds, the time-delay should provide a 
learnable brain machine interface (Olveczky et al. 2005), (Tumer \& 
Brainard 2007), (Sakata \& Brainard 2008), (Sober \& Brainard 2009). 
However it remains to be seen whether the timing jitter in the current
 system is acceptable for brain machine interface experiments in a 
system as precise as the songbird. For the zebra finch, relative 
jitter between premotor commands and auditory feedback is just a few 
milliseconds. For some experiments lower latency and lower variability
 may be desirable, and with the introduction of faster frame rate 
cameras and deterministic real-time operating systems, this latency 
and latency variability will decrease.
