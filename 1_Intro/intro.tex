\chapter{Introduction}
\label{chapter:intro}
\thispagestyle{myheadings}

\section{Motivation: optical techniques}
\label{sec:history}

Optical techniques have become a powerful tool in the process of observing and manipulating
neural behavior at the systems level when combined with genetic probes that enable reading 
and writing neural 
activity at the action potential level \cite{Emiliani:2015jl}, as they provide a stable 
interface for interacting with 
cells over a large field of view. Novel genetic probes, such as voltage 
indicators, are only broadening the questions that can be answered via optical techniques by achieving
faster time resolution and revealing subthreshold activity \cite{Gong:2015is}.

Despite these strengths, optical techniques are severely constrained by light scattering. In response to the limitations of scattering, 
a number of innovative approaches have been proposed, with varying tradeoffs in terms of quality and
feasibility. Multi-photon microscopy techniques enable imaging up to 1~mm below the surface, but
require animals to be head-fixed and hence limit behavior paradigms \cite{Horton:2013gxa}. 
Attempts to adapt multi-photon microscopy to freely behaving animals have achieved neither the
stability nor the ease of implementation desired \cite{Helmchen:2001tw,Flusberg:2005tq}.

Alternatively, more superficial brain regions can either be removed \cite{Dombeck:2010jr} or circumvented
by implanting an optical probe, such as a GRIN lens \cite{Barretto:2009hk}, a microprism \cite{Andermann:2013kc}
or a communications-grade optical fiber for fiber photometry \cite{Guo:2015gu}. These techniques can be 
effective \cite{Betley:2015cn}, but damage superficial tissue and limit imaging to a planar cross-section 
of the desired brain region. 

% BACKGROUND
%	Optical techniques
%		Probes
%			Genetic targeting
%			Fluorescent probes (wavelengths, sensitivity)
%			Stimulation
%		Field of view
%		Longitudinal access
%		Awake behaving access
%	Interface technology
%		Miniature microscopes
%		Multiphoton microscopy
%		Fiber photometry
%	Accessing deep brain regions
%		Challenge: scattering
%		Muliphoton microscopy
%		Implantable optics (GRIN lenses, prisms) - connectivity, scarring
%		Tissue removal
% PROPOSED SOLUTION
%	Goals
%		Bidirectional optical access
%		High channel count
%		Longitudinal access
%		Awake behaving access