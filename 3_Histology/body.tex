\chapter{Histology of implanted optical microfibers shows consistent splaying throughout the target
region, with minimal tissue response}

\label{chapter:histology}
\thispagestyle{myheadings}

% path to figures for this chapter
% must have a trailing slash
\graphicspath{{3_Histology/Figures/}}

\section{Introduction}

Optical techniques for recording and manipulating neural activity play a 
crucial role in advancing systems neuroscience, due to the broad field of 
view, the flexibility and specificity of viral and genetic targeting, and 
the development of novel probes offering non-overlapping spectral 
bandwidth and increasing temporal resolution 
\cite{Emiliani:2015jl,Gong:2015is}.

Yet these techniques are inherently limited in all but the most 
superficial regions of the brain, given the light scattering and 
absorption of tissue. New developments are helping to extend such optical 
techniques to deeper brain regions, but necessitate tradeoffs in terms of 
either more constrained experimental paradigms or increased tissue damage. 
Three-photon microscopy at 1300~nm has enabled recording from intact brain tissue at 
depths exceeding 1~mm \cite{Horton:2013gxa,Wang:2017jp}. And while 
three-photon microscopy is not yet amenable to recording in freely 
behaving animals, head-mounted two-photon microscopes show promise, but 
are limited to depths on the order of the mean free path of near infrared 
photons in the brain, just a few hundred microns \cite{Zong:2017eg}. 
Greater penetration depths are now being achieved with red-shifted 
fluorophores \cite{Dana:2016hx}. At depths beyond the reach of three 
photon imaging, optically interfacing with deeper layers 
and non-cortical structures has relied on implanting miniature gradient 
index (GRIN) lenses or prisms 
\cite{Jung:2004kv,Barretto:2009hk,Andermann:2013kc,Cui:2013dq}, or removing 
overlying tissue \cite{Dombeck:2010jr}. Such techniques provide 
optical access, but compromise or obliterate structures adjacent to the 
area being imaged.

Implants with a cross section 
greater than 50~\si{\micro\meter} cause neuronal damage or death over 
a zone up to 100~\si{\micro\meter} from the implant 
\cite{Seymour:2007dj}; the trauma of insertion and 
motion of the implant 
after insertion trigger a range of reactions including the immune 
foreign-body response to non-organic material, the disruption of 
oxygenation due to vessel damage, the breakdown in the blood-brain 
barrier, and excitotoxic cell death associated with accumulation of 
extracellular glutamate 
\cite{Szarowski:2003cz,Polikov:2005cq,McConnell:2009hr,Freire:2011gl}. 
Due to dense local connectivity, tissue damaged by the implant and 
foreign body response could impact network dynamics in the imaging 
plane \cite{Hayn:2015ew,Hayn:2017kj,GossVarley:2017kf}. Yet 
implants with a cross section less than 10~\si{\micro\meter} have a substantially 
diminished tissue response 
\cite{Seymour:2007dj,JohnPSeymour:2006td,Kozai:2012bp,Patel:2018cr}.

We propose a new technique to 
optically address deep brain regions through sub-10~\si{\micro\meter} 
implants. The basic idea is to cut commercially 
available leached fiber bundles \cite{Gerstner:2004to}, 
revealing the dissociated fibers, which can then be implanted. 
In this process, hundreds 
or thousands of multimode optical microfibers each with a diameter 
as small as 6.8~\si{\micro\meter} are implanted into the brain while the 
backend of the device provides optical access to the 
fibers. During the implant process, each fiber travels 
independently and finds a path of least resistance causing the 
implanted bundle to spread gradually. The small diameter of the 
fibers minimizes tissue displacement 
and decreases the likelihood both of evoking a tissue response and, 
as a result, of disrupting local network dynamics in the imaging plane.

In the approach described here, each fiber has a core and a cladding. 
The refractive index mismatch achieves near 
total internal reflection of light, enabling each fiber to interface with 
tissue near its aperture. Because of the 
splaying during insertion, the fibers will not maintain a strictly organized 
spatial mapping. Yet each fiber may provide a bidirectional interface 
with a small volume of tissue near the tip of the fiber, and potential 
correlations across fibers can enable reconstructing a relative spatial 
topography. Outside of the brain, the fibers converge to a polished 
imaging surface, where each fiber is arrayed in a tightly packed 
lattice that can interface with a traditional fluorescence microscope. 

Based on histology and immunohistochemistry, we demonstrate that the 
bundles of optical microfibers splay during insertion into the brain, 
achieving a spatially distributed set of fibers throughout the target 
brain region. The small cross section of the individual fibers displaces 
less tissue than GRIN lenses (for example, a bundle of 2,000 fibers 
displaces half the volume of a 500~\si{\micro\meter} diameter lens), 
and hence may preserve more neurons and promote more natural network 
dynamics in the target region. Based on simulations 
of the optical profile of individual fibers, we assess the sensitivity of 
the fibers as a multi-channel, bidirectional optical interface. Finally, 
we show that fluorescence signals can be recorded from diffusing 
fluorescent beads through these small-diameter optical microfibers.

\section{Methods}
\label{sec:methods}

\subsection{Fibers}
\label{sec:methods-fibers}

The fibers we use are leached fiber bundles produced as flexible medical 
endoscopes. This work primarily relied on bundles of 4,500 fibers where 
each individual fiber has a diameter of 8~\si{\micro\meter} (Schott 1534180), although 
variations exist in the number of fibers (3,500--18,000) and the diameter 
(6.4--11.9~\si{\micro\meter}). The bundles are built for coherent imaging and constructed 
from three types of glass, a core (diameter: 5.1~\si{\micro\meter}, refractive index: 
1.605), a cladding (thickness: 1~\si{\micro\meter}, refractive index: 1.56) and an 
acid soluble glass (thickness: 0.4~\si{\micro\meter}). The bundles are manufactured 
as traditional coherent 
fibers, and then the acid soluble glass is dissolved for bundle 
flexibility \cite{Gerstner:2004to}. The ends of the fibers come together 
in polished imaging surfaces held in ferrules. The dissociated fibers are covered 
in a flexible silicone sheathing.

We cut the bundles in half, using a scalpel or razor, sacrificing the spatial 
cohesion, but providing access to the individual, dissociated fibers. The silicone 
sheathing was cut back to expose the fibers, and we then cut a fraction of the 
exposed fibers to reduce the implant size to a target number of fibers (varied 
over implants to assess tissue impact). The remaining fibers were secured 
together by forming a bead of light-cured acrylic (Flow-It ALC, Pentron Clinical) 
around the fibers, leaving 4--5~mm of fibers exposed (\Fref{fig:bundle}b). The exposed fibers could be 
further shaped using fine scissors, creating a bevel. Such pre-implant shaping 
increases the distribution of depths of the fiber tips. At this point, the 
dissociated fibers can be directly implanted into brain tissue; the other end of 
the fiber, containing the ferrule and polished surface with fibers aligned and 
arranged in a lattice, can be readily interfaced with a fluorescence microscope or 
other optical configuration (\Fref{fig:bundle}).


\begin{figure}
\includegraphics[width=\textwidth]{figure_bundle.png}
\caption[Diagram of bundle of microfibers as an 
optical interface.]{\textbf{Bundles of microfibers as a potential deep brain 
optical interface.} (a) The polished imaging surface is 
mounted in a traditional fluorescence microscope, while 
individual fibers with a diameter as small as 
6.8~\si{\micro\meter} are implanted into the brain. (b) The polished 
imaging surface that connects with the microscope. 
(c) A bundle of 18,000 fibers. (d) Light propagates with 
near total internal reflection, allowing it to deliver 
and collect light at the tips of the fibers. Six fibers are shown in a 
fluorescein solution, with pink lines added to emphasize 
fiber path.}
\label{fig:bundle}
\end{figure}

Optical attenuation of the fibers was measured to be 3.38~$\pm$~0.03~dB (std. dev.) 
for a 840~mm long bundle (4~dB/m). Attenuation was measured using collimated light in the 
446--486~nm range (relevant for exciting GFP-based indicators, such as GCaMP, or 
stimulating channelrhodopsin) focused on the polished imaging surface using a lens 
with numerical aperture matched to the fibers. In this measurement, we assumed that
19.4\% of the incident light enters the cores based on the surface area of the fiber 
bundle, the fiber count, and the core diameter. 
After cutting the bundle, attenuation for the 420~mm bundle measured 
from the splaying fibers is 3.78~$\pm$~0.02~dB (std. dev.), indicating that most of 
the light carried by the bundle is transmitted from the cut ends.

\subsection{Histology}

\paragraph{Animals.} Animal care and experimental procedures were 
approved by the Institutional Animal Care and Use Committee (IACUC) 
of Boston University (protocols 14-028 and 14-029). Fibers were 
implanted in 27 adult zebra finches ($>$ 120 days post hatch). Of 
the animals, eight were implanted with alternative fibers 
(different materials and different fiber diameters; results not 
shown). Of the remaining nineteen, fifteen were used for histology 
described in this paper with four animals being excluded due to 
poor slicing (tearing of the tissue when slicing through the 
fibers) or poor staining (during immunohistochemistry).

\paragraph{Fiber implant.} Anesthesia was induced with 4\% isoflurane 
and maintained at 1--2\% for the duration of the surgery. An analgesic 
(0.5~mg/kg meloxicam, Eloxiject) was injected intramuscularly into the 
breast at the start of the procedure. The animal was placed in a 
stereotaxic instrument and feathers were removed from the scalp. The 
scalp was cleaned with Betadine and ethanol. A local anesthetic 
(4~mg/kg bupivacaine) was injected subcutaneously into the scalp, and 
an incision was made along the anterior-posterior axis.

The skull over the implant point (area X) was localized based on head 
angle (\ang{20}) and stereotactic coordinates (5.8~mm anterior, 1.5~mm 
lateral). In order to accommodate the bundle of fibers, a 0.5--1~mm 
diameter craniotomy was created, with the size matched to the bundle. 
The craniotomy was created by first using a dental drill to remove the 
outer layer of bone, then by using an ophthalmic scalpel to remove the 
inner layer of bone \cite{Long:2010db}. The dura within the craniotomy 
was removed using either a dura pick constructed from sharpened 
tungsten or an ophthalmic scalpel.

The fiber bundle was prepared by securing the fibers together in a 
bead of light-cured 
dental acrylic (Flow-It ALC, Pentron Clinical) and cut to 3--5~mm. Using 
a digital manipulator attached to the stereotaxic rig, the fiber bundle 
was positioned over the durotomy and slowly lowered into the tissue at 
a rate of approximately 500~\si{\micro\meter} per minute. The insertion rate varied 
based on the number of fibers and visual inspection of the tissue 
surrounding the implant. Larger implants (more than 250 fibers) could 
result in a noticeable depression or ``dimpling'' in the tissue before 
the bundle passed through the surface of the brain. Such dimpling was 
generally observed during the first 250--350~\si{\micro\meter} of 
insertion; beyond that depth, the size of the depression remained 
consistent as we continued to lower the implant. We found that the 
visible dimpling could be alleviated by lowering the implant an 
additional 50~\si{\micro\meter} past the desired depth, waiting for five 
minutes, then returning the implant to the desired depth. We did not 
observe bleeding 
associated with the implant or the dimpling. After the fibers 
were lowered to a depth of 2.7--2.9~mm (measured from the point when the 
fibers enter the tissue), additional light-cured dental acrylic was used 
to secure the fiber bundle to the skull surrounding the craniotomy. 

Animals received nonsteroidal anti-inflammatories (0.5~mg/kg meloxicam) 
both before the surgery via injection (Eloxiject) and after the surgery 
in their food (Metacam), as well as topical antibiotics 
(Pfizer Terramycin) after the surgery.

Three days post implant, animals were returned to the aviary and housed 
socially. Animals used to image the distribution of fibers were perfused 
after 21 to 331 days (mean 88 days). Animals used for 
immunohistochemistry staining were perfused after 77 to 395 days 
(mean 176 days).

\paragraph{Animal perfusion and fixation.} Animals were injected 
with 0.1~\si{\milli\liter} 10\% 
sodium pentobarbital intramuscularly. Once anesthetized, the animals 
were perfused intracardially with phosphate-buffered saline (PBS) 
followed by 4\% paraformaldehyde in 0.1~\si{\Molar} PBS. The skull and brain 
were separated from the body. Leaving the skull in place (as the 
fibers are anchored to the skull), small cracks were made in the bone 
to ensure penetration of the fixative. The skull and brain was 
immersed in 4\% paraformaldehyde in 0.1~\si{\Molar} PBS overnight. Next, as 
cryoprotection, it was immersed in 15\% sucrose in 0.1~\si{\Molar} PBS overnight, 
followed by 30\% sucrose in 0.1~\si{\Molar} PBS for a second night. Placing the 
skull upside down such that the implant trajectory was roughly 
perpendicular to the mounting slide, the skull was frozen (-20\si{\celsius}) 
in embedding medium (Optimal Cutting Temperature Compound, Tissue-Tek) 
for 30 minutes and sectioned 
in a cryostat (Leica CM3050S, with Thermo Scientific MB22 microtome 
blades) in either 70 or 100~\si{\micro\meter} thick slices, 
cutting through the skull and perpendicular to the fiber bundle implant. 
Due to the thin, pneumatized bone of the songbird, cryosectioning 
through the skull was possible without any decalcifying process. 
For optimal cutting, blades were regularly shifted and replaced 
to ensure a fresh cutting surface was always in use; without such 
precautions, the worn blades were more likely to catch on fibers 
and tear surrounding tissue. 
Some sections were discarded because of tearing. Slices were 
either mounted on slides or were transferred to wells 
containing PBS and processed for immunohistochemical staining 
as described below.

\paragraph{Histology.} To quantify the splay of fibers, brightfield 
microscopy images were collected of slices mounted on slides and 
secured with coverslips. Images were collected from slices at various 
depths.

\paragraph{Immunohistochemistry.} In order to assess tissue health and 
imaging viability, a selection of slices taken at various depths were 
processed to label neurons via NeuN antibodies. Slices were washed in 
PBS, then in 0.3\% Triton X-100 in PBS for 30 minutes and finally in 
a solution of 0.3\% Triton X-100 and 5\% normal donkey serum (NDS) 
for 45 minutes. The slices were then placed in a solution of the 
primary antibody (MAB377 Anti-NeuN, 1:500, EMD Millipore) made with 
3\% bovine serum albumin (BSA) and 0.3\% Triton X-100 in PBS. The 
wells were placed on a rotator and allowed to incubate at 4\si{\celsius}
overnight. Slices were washed in PBS ($\times$3, 10 minutes each). Next, the 
slices were placed in a solution of the secondary antibody 
(715-025-150 Rhodamine [TRITC] AffiniPure Donkey Anti-Mouse IgG, 1:500, 
Jackson ImmunoResearch). The wells were again placed on a rotator and 
allowed to incubate at 4\si{\celsius} for one hour. Slices were washed in 
PBS ($\times$3, 10 minutes each). Next, 1~\si{\milli\liter} of DAPI stain 
(4',6-Diamidino-2-Phenylindole, Dihydrochloride, 300~\si{\micro\Molar} solution, 
1:1000, D1306, Thermo Fisher Scientific) was added to each well. After 
three minutes, the slices underwent a final wash ($\times$2, 5 minutes), 
before being mounted on glass slides with an anti-fading mounting 
medium (Fluoro-Gel, EMS) and secured with a coverslip. Some 
immunohistochemistry samples were not usable, due to ineffective 
staining or fibers becoming dislodged during the washing process.

\paragraph{Microscopy.} Slices were imaged using an upright fluorescence 
microscope (Nikon Eclipse NiE, with a DS-Qi1 Monochrome camera and 
controlled by NIS-Elements: Advanced Research), illuminated by an LED 
light source (SOLA Light Engine). To assess splay, we 
used either a 4$\times$ (Plan Fluor, NA 0.13) or a 10$\times$ (Plan Fluor, NA 0.3) 
objective. To image immunohistochemistry, we used a 20$\times$ objective 
(Plan Apo Lambda, NA 0.75).

\paragraph{Qualitative and quantitative analysis.} To quantify fiber 
splay, brightfield images were collected from slices near the tip of 
the fiber. Fibers were manually annotated using a custom MATLAB 
program for organizing and analyzing histology. To calculate a measure 
indicative of the splay of the fibers, a bivariate normal distribution 
was fit to the position of the fibers in the slice and the area of the 
ellipse representing two standard deviations of the distribution (the 
95\% confidence interval) was calculated. The data presented are from 
11 animals, reflecting the animals implanted with bundles consisting 
of 7--8~\si{\micro\meter} diameter fibers with at least a three week recovery 
period and where the tissue at the tip of the implant was cleanly sliced 
(see note above about sectioning).

To quantify the presence of neurons in proximity to fibers, two-channel 
fluorescence (with NeuN in red and DAPI in blue) and brightfield 
images were collected from the target implant region (area X). Control 
images were collected from the contralateral region (without an implant) 
to measure baseline neural distributions and densities. Neurons were 
manually annotated based on a consensus of the NeuN and 
DAPI signal, and fibers were manually annotated based on both the 
histology and brightfield images. For slices with fibers, the distance from each 
fiber to the nearest neuron was calculated (fibers where the edge of the 
image was closer than the nearest neuron were ignored), subtracting the 
radius of the fiber. As a control, random points were selected on the control slices 
without fibers, and the distance to the nearest neuron was calculated 
(points could be selected at or on neurons, resulting in a distance of 
zero).

In addition, NeuN-stained cell density was calculated for the 
50~\si{\micro\meter} region surrounding each implant, normalized by 
densities calculated on the control slices. To account for the close 
proximity of neighboring fibers, the cross sectional area of neighboring fibers 
was subtracted from the area of the 50~\si{\micro\meter} region when calculating density surrounding 
implants. The data presented are based on twelve annotated slices from 
five animals, reflecting all animals implanted with bundles of 
7--8~\si{\micro\meter} diameter fibers with at least a ten week recovery 
period and successful immunohistochemical staining.

\subsection{Modeling}
\label{sec:methods-modeling}

\paragraph{Fiber profile.} The optical profile for a single fiber was 
generated via a Monte Carlo simulation of 10,000,000 
photon packets traveling through a 1~mm$^3$ volume (modeled as isotropic 
5~\si{\micro\meter} voxels) \cite{Boas:2002ue}. Photon packets enter the tissue at 
[500~\si{\micro\meter}, 500~\si{\micro\meter}, 200~\si{\micro\meter}] with a Gaussian distribution 
reflecting the NA of the fiber (0.377). Within each voxel, the photon 
packet can be scattered ($\mu_s$ = 10~mm$^{-1}$ with anisotropy g = 0.9 
\cite{Yi:2012wp}) or fractionally absorbed ($\mu_a$ = 0.337~mm$^{-1}$ 
for 490~nm light, $\mu_a$ = 0.343~mm$^{-1}$ for 512~nm light based on 
3\% blood volume fraction [BVf] \cite{S:2010hi}, 15~g/DL hemoglobin 
concentration \cite{Raabe:2011uo}, an oxygenation fraction of 70\% and 
extinction coefficients for hemoglobin \cite{kollias1999tabulated}). 
The 3D path of each photon packet is averaged together, normalized and 
visualized as a 2D slice through the volume. The fluorescence signals 
received by individual fibers, given the illumination profile from the 
superposition of the optical profiles emitted from all of the fibers, 
is calculated following the procedure described in 
\cite{Hillman:2004wca,Burgess:2008uu}.

\paragraph{Neural interface simulation.} To simulate interfacing with 
a neural population, a 1.2~mm$^3$ volume of tissue was modeled. This 
volume is consistent with area X in the adult zebra finch 
\cite{Bottjer:1985vs} and is illustrative of a deep brain region. 
A target subpopulation of neurons of interest is modeled as 
uniformly distributed through the volume with a density of 780,000 
neurons per mm$^3$, based on the density of medium spiny neurons 
in area X in male zebra finch that are one year old 
\cite{KosubekLanger:2017jb}. All cells in the target subpopulation 
are assumed to express the relevant genetic probe.

Based on the histological data on splaying, the fiber bundle is 
assumed to have a bivariate normal distribution in xy space with 
standard deviation ($\sigma$) based on the number of fibers in the 
bundle. The fiber depth will vary based on preparation of the 
bundle (how the fibers are cut prior to implant) and the path of 
splay; this variability is modeled as a normal distribution of 
depths with standard deviation $\sigma$ = 30~\si{\micro\meter}.

The strength of stimulation or excitation for individual neurons 
is calculated for each fiber by identifying the sensitivity of 
the voxel that corresponds with the position of the neuron relative 
to the tip of the fiber. The per fiber optical intensities are summed 
across all fibers in the bundle to calculate the total potential 
stimulation/excitation strength. These 
values are normalized as a percentage of maximum fluence in the tissue.

To evaluate the ability to uniquely address neurons through 
illuminating a subset of $n$ fibers, a 20,000 iteration Monte 
Carlo simulation is used to select 
random permutations of $n$ fibers. For each iteration, the 
number of neurons activated by the cumulative optical power of the 
selected fibers is compared with the number of neurons 
activated if each fiber was illuminated independently.

The round-trip fluorescence yield for pairs of fibers and neurons, 
a measure of expected fluorescent emission collected by the fiber from the neuron, is 
calculated by multiplying the total excitation strength for the 
neuron (as described above) by the sensitivity of the voxel that 
corresponds with the position of the neuron relative to the tip of 
the fiber (representing the time reversal of emission from the 
neuron reaching the fiber tip) \cite{Hillman:2004wca,Burgess:2008uu}. 
This round trip fluorescent yield is normalized based on the maximum 
possible yield.

\subsection{Fluorescent beads}

To validate the recording capability of the fiber bundles, the tips of loose 
fibers were immersed in a solution of water and fluorescent beads 
(Bangs Laboratories FSDG007, 7.32~\si{\micro\meter} diameter, 480~nm excitation, 
520~nm emission). The ferrule and polished imaging surface were held 
below a traditional fluorescent microscope (Olympus, 20$\times$ objective) 
with a broadband white LED (Thorlabs SOLIS-3C) set at 60\% brightness and 
a GFP filter cube (Semrock BrightLine GFP-4050B, 466/40 excitation, 525/50 
emission, 495 dichroic). Excitation power from the objective was measured 
at 6.27~mW. As beads diffused in the water, changes in fluorescence were 
recorded by a sCMOS camera 
(Hamamatsu ORCA-Flash4.0 v2) with a resolution of 2048$\times$2048 16 bit 
pixels and an exposure of 50~ms per frame. Saved CXD files were 
processed in MATLAB using a custom pipeline. Frames were motion corrected 
using the Scale-Invariant Feature Transform (SIFT) algorithm 
\cite{vedaldi08vlfeat,lowe1999object,Lowe:2004kp}. A standard 
deviation image created by calculating the standard deviation of 
pixels across frames was used to identify those fibers that were in the 
solution and where bead diffusion resulted in variability in the 
fluorescence. For the identified fibers, traces were generated by 
extracting and averaging all pixels that corresponded with the fiber. 
Traces were converted to $\Delta$F/F$_0$, where F$_0$ corresponds with 
the 5th percentile intensity (i.e., background intensity when there is 
minimal fluorescence from nearby beads).

To calculate the contrast-to-noise ratio (CNR) for the bead recording, 
we performed a second recording to measure noise. The fibers were placed 
in a solution of fluorescein and water, such that the fiber brightness 
matched the peak brightness observed during the fluorescent bead recording. 
The signal was recorded, and again, traces were generated by extracting 
and averaging all pixels that correspond with each fiber. For the CNR, we 
calculate the contrast from the fluorescent bead recording by subtracting 
the 5th percentile from the 95th percentile intensity and averaging across 
fibers; we calculate the noise as the standard deviation for traces from 
the fluorescein recording.

\section{Results}

\subsection{Histology}
\label{sec:results-histology}

Bundles of between 50 and 5,000 microfibers were implanted into 
zebra finch basal ganglia (area X) at a depth of 2.9~mm. To 
understand the impact of the bundles, histologic samples 
were collected to measure the distribution of 
fibers in tissue and to evaluate the distance been fiber tips and 
the nearest NeuN-stained neurons.

With the fibers anchored to the intact skull, the tissue was 
fixed and cryosectioned perpendicularly to the implant penetration 
angle. Sections were imaged and annotated to record the spatial 
distribution of microfibers at different depths. During insertion, 
each fiber follows a path of least resistance, splaying through 
the brain tissue. In these perpendicular sliced sections, the 
distribution of fibers resembles a bivariate normal distribution 
throughout the target region. In \Fref{fig:histology_implant}, 530 fibers can 
be seen distributed spanning over 1~mm of tissue, while only 
displacing a cross sectional area of 26,640~\si{\micro\meter}$^2$; a 
1~mm diameter GRIN lens to access the same region would have a 
cross sectional area of 785,398~\si{\micro\meter}$^2$.

\begin{figure}
\includegraphics[width=\textwidth]{figure_histology_implant.png}
\caption[Histology showing microfibers splayed 
throughout target region.]{\textbf{Histology at tip of implant shows 
microfibers splayed throughout the target region.} 
A 100~\si{\micro\meter} thick brain section showing the tips of a 
bundle of 530 optical microfibers implanted at a 
depth of 2.95~mm. Before insertion, the bundle had a 
diameter of 570~\si{\micro\meter}. This section was 
collected four months after implant, and the brain sectioned 
perpendicularly to the insertion angle. The cross sectional  
area of tissue displaced by the microfibers 
(annotated in green) is 26,640~\si{\micro\meter}$^2$ 
(pink circle).}
\label{fig:histology_implant}
\end{figure}

Implant conditions account for much of the variability in 
the spread of the fibers. Based on anecdotal observations, 
the configuration of the fibers prior 
to implant---specifically, the spatial arrangement of fibers 
in the acrylic anchor point (used both to hold the fibers 
during the implant and the to anchor the fibers to the 
skull), and the spread of the fibers below this acrylic 
anchor point---appears to affect the final distribution of 
the fibers. For example, if the fibers spread in the air 
before coming into contact with the tissue, we tended to 
observe greater spread after insertion into the tissue. 
The configuration of the fibers in the acrylic anchor point 
is difficult to control, as we sought to avoid directly 
squeezing or stressing the fibers. But we found that we 
could influence the amount of spread below the anchor point 
by keeping the fibers dry; if the fibers get wet, there is 
greater adhesion during insertion and, as a result, a more 
narrow distribution in the tissue. As a result, we avoided 
wetting the fibers and minimized moisture on the surface of 
the tissue prior to implant (as that would get wicked into 
the fiber bundle and increase adhesion).

In \Fref{fig:histology_depth}, the distribution of the microfibers in 
the tissue can be seen to increase over the four slices from 
different depths in the same animal; the splay area is calculated by drawing a 
bounding ellipse containing 95\% of the fibers. For each 1~mm of 
implant depth, the diameter of the splay area increases by 
229.1~$\pm$~51.1~\si{\micro\meter} (std. dev., 
based on 9 pairs of slices from 5 animals); see 
\Fref{fig:plot_diameter}.

\begin{figure}
\includegraphics[width=\textwidth]{figure_plot_diameter.eps}
\caption[Diameter of fiber splay as a function of fiber count 
and depth.]{\textbf{Diameter of splaying increases linearly with 
depth.} Left: Splay of fibers at a depth of 2.9~mm, in the 
target region of songbird basal ganglia from 11 animals. The plot shows the 
diameter of the ellipse describing the splay of the fibers 
for various implant sizes. As the number of fibers increases, 
the area accessed by the fibers increases. Right: For five 
animals, slices were collected at multiple depths to estimate 
splay diameter as a function of depth. For each 1~mm of 
implant depth, the diameter of the splay increases by 
229.1~$\pm$~51.1~\si{\micro\meter} (std. dev.).}
\label{fig:plot_diameter}
\end{figure}

\begin{figure}
\includegraphics[width=\textwidth]{figure_histology_depth.png}
\caption[Histology showing microfibers at different 
implant depths.]{\textbf{Histology at different depths as the 
fibers splay during insertion.} A bundle of approximately 
1,125 optical microfibers implanted at a depth of 2.95~mm. 
Eight weeks after the implant, the animal was perfused 
and the brain sectioned perpendicularly to 
the insertion angle. At the surface, the bundle diameter 
was 1.03~mm. These 70~\si{\micro\meter} thick slices from depths 
(a) 2.76~mm, (b) 2.34~mm, (c) 1.57~mm and (d) 0.59~mm 
reveal a gradual spreading of the optical fibers during insertion as each fiber 
follows a path of least resistance.}
\label{fig:histology_depth}
\end{figure}

Tissue sections from animals with chronic implants 
(10+ weeks post implant) underwent NeuN staining to label 
neurons and DAPI staining to label nuclei. Since the red blood 
cells of birds contain DNA, DAPI labelled cells that are not 
NeuN stained include populations of glia, astrocytes, red blood 
cells, and any other non-NeuN stained cell nuclei. The slices 
show NeuN-stained neurons in close proximity to the fibers (see 
\Fref{fig:histology_neun}). In instances where two or more fibers remain close 
during insertion, the proximity of the fibers may adversely 
affect the immediate tissue, as suggested by an increased 
presence of non-neural cells (DAPI stained but not NeuN stained) 
around such ``clumps'' of fibers.

\begin{figure}
\includegraphics[width=\textwidth]{figure_histology_neun.png}
\caption[NeuN and DAPI staining of tissue at fiber 
implant tips.]{\textbf{At chronic time points neurons are found in 
close proximity to fibers.} Three sections from zebra finches implanted with 
optical microfibers, collected at least ten weeks post-implant. 
Sections are near the tip of the implant, within the basal ganglia 
(area X, depth 2.9~mm). (a) and (b) show implants in the basal 
ganglia (bundle sizes of 4,500 and 1,125 fibers respectively), 
(c) shows unimplanted basal ganglia, and (d) shows a 
corresponding brightfield image used to confirm fiber locations. 
Red is NeuN (neurons), blue 
is DAPI (nuclei) and green dots are manual annotations that 
reveal fiber locations. 
The immunohistochemistry shows NeuN-stained cells in close 
proximity to fibers. In some cases, we observe a dense 
circle of DAPI stained cells in close proximity to the 
fibers (arrow), suggesting either bleeding (in birds, red 
blood cells have DNA) or a reactive tissue response (such as glia 
or astrocytes). This most frequently occurs at locations 
where multiple fibers are in close proximity (this can 
occur with bundles of over 4,000 fibers or when fibers are 
wet prior to insertion). The length scale of reactive 
tissue response is approximately an order of magnitude 
smaller than for silicon electrode shanks with a 50~\si{\micro\meter} 
profile \cite{Szarowski:2003cz}.}
\label{fig:histology_neun}
\end{figure}

By annotating both the fibers and the neurons, the presence of 
NeuN-stained neurons near the fibers can be compared to control 
slices (same region, no implant) to evaluate tissue impact. 
\Fref{fig:plot_neuron} compares the distance 
from fibers to neurons in implant slices to the distance 
between randomly selected points and neurons in control slices. 
The control measurement provides a lower bound for distance to 
the nearest neuron, if the implant had no impact on the tissue. 
For the implanted slices, the distance from a fiber to the nearest 
NeuN-stained neuron is on average 12.81~$\pm$~9.22~\si{\micro\meter} 
(std. dev.), while on the control slice, 
the distance from a randomly selected point to the nearest neuron 
is on average 8.32~$\pm$~4.72~\si{\micro\meter} (std. dev.).

\begin{figure}
\includegraphics[width=\textwidth]{figure_plot_neuron.eps}
\caption[Distribution of distance from fiber to 
NeuN-stained cell.]{\textbf{Histology reveals minimal tissue damage.} 
The distribution of distances to the nearest NeuN-stained 
cell for implants (measuring from the edge of each fiber) 
and unimplanted controls (measuring from randomly selected 
points). The NeuN and DAPI staining shows that there are 
intact neurons in close proximity to the fibers; 
85\% of the fibers have a neuron within 20~\si{\micro\meter}.}
\label{fig:plot_neuron}
\end{figure}

We also can compare the NeuN-stained cell density surrounding each 
fiber relative to the cell density seen in the control slices. In 
the 50~\si{\micro\meter} region surrounding each fiber, we observe a NeuN density 
of 69.8\% $\pm$ 17.9 (std. dev.) the density seen in control 
slices (same region, no implant). Because the 50~\si{\micro\meter} 
surround typically includes other fibers, we subtract the cross 
sectional area of such neighboring fibers from the 50~\si{\micro\meter} 
area when calculating the density.

NeuN staining alone does not provide a comprehensive evaluation 
of tissue or neural health; variability in staining does not 
consistently indicate differences in neural populations and does 
not capture non-neuronal changes in tissue health 
\cite{UnalCevik:2004cp,Collombet:2006fj,Duan:2015iq}. Despite 
having a narrow immunohistochemical tool to evaluate tissue 
health, our histology data are consistent with the possibility 
that circuits remain healthy in the vicinity of the fiber tips.

\subsection{Modeling}
\label{sec:histology-modeling}

To quantify the potential neural population accessible via the 
optical microfibers, we modeled the optical profile of a single 
fiber and a bundle of fibers throughout a volume of tissue 
(see \sref{sec:methods-modeling}) \cite{Boas:2002ue}.

\Fref{fig:profile} shows the normalized optical profile for a single fiber 
in tissue with spatially uniform anisotropy, scattering and 
absorption coefficients based on brain tissue measurements. At 
a distance of 40~\si{\micro\meter} from the tip of the fiber, the number of photon 
packets passing through an arbitrary point 
in the tissue drops below 10\%. Although the fiber can weakly interface with a 
larger volume of tissue due to the scattering of light in the 
brain, individual fiber fluorescence will be dominated 
by neurons within 40~\si{\micro\meter} of the tip of the fiber. The viability of 
recording fluorescent signals depends on a number of additional 
properties that will vary based on the animal model and target 
region, including the indicator brightness, specificity of 
expression, density of the neural signal, and tissue 
autofluorescence.

\begin{figure}
\includegraphics[width=\textwidth]{figure_profile.eps}
\caption[Profile of single fiber.]{\textbf{A single fiber would primarily interface with 
neurons in close proximity to the tip, based on the tissue 
scattering and absorption.} The normalized log intensity 
emission profile of an optical microfiber with tip 
positioned at [0, 0]. The profile is a Monte Carlo 
simulation of photon packets propagating through 
brain tissue, with scattering and absorption properties 
estimated for 490~nm light. The simulated profile shows 
a strong interaction with tissue immediately below the 
tip of the fiber, enabling localized photometry or 
stimulation; the weak interactions with a larger volume 
of tissue will contribute background in recordings, or 
delocalized optogenetic excitation.}
\label{fig:profile}
\end{figure}

Similarly, stimulation through the fiber will most strongly 
modulate neural activity within the region immediately 
surrounding the tip of the fibers. Based on the coupling 
2.5~\si{\micro\watt} of 470~nm light into each fiber, and accounting for 
attenuation measurements of the optical path and fibers, and 
the simulated optical profile, one fiber will provide sufficient 
optical power to activate channelrhodopsins in a 18,000~\si{\micro\meter}$^3$ 
region surrounding the tip of the fiber given a 5~mW/mm$^2$ 
activation threshold \cite{Yizhar:2011jv}. For the modeled neural 
subpopulation (medium spiny neurons in the zebra finch basal 
ganglia, with a density of 780,000 neurons per mm$^3$ 
\cite{KosubekLanger:2017jb}), this stimulation region equates 
to activating approximately 14 neurons. 
For comparison, this stimulation region equates to activating 
approximately 5 neurons in mouse hippocampus CA1, based on a 
density of 275,000 neurons per mm$^3$ 
\cite{AyberkKurt:2004jg,Richards:2013fe}.

Based on the histology of splaying fibers described above, it is 
possible to overlay profiles for hundreds or thousands of fibers 
throughout a brain region to quantify the properties of the 
bundle as an interface.

\Fref{fig:model} shows a distribution of normalized 
excitation/stimulation power reaching neurons for a simulated bundle of 
500 fibers. Although the neurons receiving the most optical 
power are within the first 100~\si{\micro\meter} below the mean implant depth, the 
scattering properties of the tissue and the overlap in the 
excitation profile of fibers means that the excitation 
light will affect many more cells 400--600~\si{\micro\meter} 
below the implant depth.

\begin{figure}
\includegraphics[width=\textwidth]{figure_model.eps}
\caption[Distribution of neurons accessible via fiber 
bundle.]{\textbf{A computational model indicates that 
optical microfibers would record or 
stimulate neurons immediately below the fiber tips.} 
Left: Distribution of light intensities reaching all 
modeled neurons, for uniform illumination of all fibers 
in the fiber bundle. These values are normalized by the 
maximum possible optical power (i.e., the power at the 
point in the tissue with the highest intensity). Blue 
dots are individual neurons, and the red line is a 
depth distribution of neurons that receive $>$1\% of max 
excitation, indicating that for full bundle illumination, 
optical stimulation would activate cells far away from 
the fiber tips. Right: The round-trip fluorescence yield 
is calculated by first taking the total excitation power 
reaching the neuron (left) and scaling that by the 
strongest overlapping fiber profile (representing the 
collected fluorescence emission). These values are 
normalized by the maximum possible round-trip 
fluorescence yield (the maximum achievable given the 
excitation profile). Blue dots are individual neurons, 
and the red line is a depth distribution of neurons with 
$>$1\% of max excitation.}
\label{fig:model}
\end{figure}

In stimulation experiments, rather than illuminating all fibers, 
a subset of the fibers can be illuminated to produce more 
precisely targeted cellular modulation. 
Given the splay of the fibers, the vast majority of fibers 
can address a unique set of cells closest to the tip; yet 
the scattering properties of the tissue and the overlaps in 
the profiles mean that delivering stimulation through 
multiple fibers will increase activation in deeper regions 
and at the overlap between fiber profiles. For example, 
our model suggests that activating each fiber independently 
at non-overlapping times in a bundle of 500 fibers would 
serially stimulate approximately 4,600 cells; if all fibers were 
active simultaneously, there would be sufficient optical 
power to stimulate approximately 93,000 cells in the modeled 
neural population.

By activating small subsets of fibers, it is possible to avoid 
broad activation, while still exploring stimulation patterns 
with many degrees of freedom. By simulating overlaps in the 
optical profile for random sets of 10 fibers in a bundle, light 
delivered through the ten fibers will only activate an average 
of 11 more neurons (9.3\%) than if the fibers were activated 
individually. Increasing the number of simultaneously active fibers will increase 
the crosstalk between the stimulation profiles. For example, 
sets of 50 fibers will activate an average of 506.4 more 
neurons (87.3\%) than if the fibers were activated individually.

To evaluate the bundle as a potential recording interface, we calculate the 
round-trip fluorescence yield, indicative of how much 
fluorescent activity is collected by each fiber. Consistent with 
the profile for a single fiber, neurons within 40~\si{\micro\meter} of the mean 
implant depth have the highest fluorescence yield for recording 
purposes; cells up to 120~\si{\micro\meter} away will contribute to the signal, yet 
low fluorescence yield will likely relegate this contribution to 
indistinguishable background.

\subsection{Fluorescent beads}

As a preliminary test of the fluorescence recording capability, 
we immersed dissociated fibers in a suspension of fluorescent 
beads in water and recorded fluorescence traces as the beads diffused 
through the sensitivity profile of the individual fibers. Taking a standard deviation 
of pixel intensities over the recording, we generated a standard 
deviation image of the polished imaging surface, which revealed 
those fibers with large fluctuations in measured fluorescence 
resulting from the diffusing beads (\Fref{fig:beads}). Extracted traces (average 
intensity for pixels corresponding with the fiber), shown 
in \Fref{fig:beads}, reveal minimal crosstalk between neighboring 
fibers and a high signal-to-noise ratio. With excitation power of 
6.27~mW measured at the imaging surface of the fiber bundle, we 
observed fluctuations in fluorescence intensity up to 
23.7$\times$ the F$_0$ intensity.

We calculate a CNR (contrast-to-noise ratio) with contrast 
25.88$\times$ the standard deviation of the noise observed during 
a similar recording with the fibers immersed in a uniform 
fluorescein solution (with fluorescent brightness matched to the 
peak signal in the bead recordings).

\begin{figure}
\includegraphics[width=\textwidth]{figure_beads.eps}
\caption[Recording of diffusing fluorescent beads 
via fiber bundle.]{\textbf{Recording of diffusing fluorescent 
beads reveals minimal cross-talk between neighboring 
fibers.} The dissociated 
end of a bundle of fibers was immersed in a 
suspension of fluorescent beads in water, while the 
imaging surface was recorded via a traditional 
fluorescence microscope. The image on the left is 
a standard deviation image corresponding with a 
1024$\times$1024 portion of the full sCMOS sensor, 
where pixel brightness corresponds with variability 
over the 5 second recording. It accentuates those 
fibers immersed in water and with diffusing beads 
in close proximity to the fiber tips. Traces from 
a selection of fibers (circled) were extracted to 
show intensity over the recording. Fiber intensity 
varies as beads diffuse around the tips of the 
fibers, showing a high signal-to-noise ratio and 
minimal crosstalk between neighboring fibers.}
\label{fig:beads}
\end{figure}

\section{Discussion}

Our histological results demonstrate that bundles of optical 
microfibers may provide an alternative to GRIN lenses to 
optically address 3D volumes in deep brain areas. The 
fibers self-splay during the implant process, achieving 
a distribution that resembles a bivariate normal distribution, 
with the diameter frequently exceeding 1~mm at an implant 
depth of 2.9~mm. There appears to be a relationship between 
the number of fibers and the diameter of the splay, but the 
trend does not achieve significance in the data set 
(r$^2$ = 0.5, p = 0.11). Small bundles show increased 
variability in splay that requires further exploration. We 
believe that implant conditions, such as the distribution of 
the fibers before entering the tissue, may account for much 
of this variability. In addition, as visible in 
\Fref{fig:histology_depth}, the final distribution is not 
always symmetric, which could reflect tissue heterogeneity, 
non-perpendicular implant angles or non-perpendicular 
sectioning of the tissue. 

At the fiber tips, we find NeuN-stained cell bodies in close 
proximity to the fiber, suggesting the small diameter, 
flexible microfibers may evoke a smaller foreign body response 
than that seen with larger glass or electrode implants 
\cite{McConnell:2009hr,Lee:2016ko}, but more extensive 
histological analysis is needed in a range of species. Because 
our current histology is unable to identify neurons in front of 
(below) the fiber apertures, we instead rely on the perpendicular 
sections to identify stained neurons adjacent to the fibers in 
the section closest to the tip of the implant, providing an 
estimate of the presence of neural signals within range of the 
fiber sensitivity profile. Future experiments may be able to 
reconstruct the full 3D path of the fibers through the tissue, 
and as a result, the presence of neurons within the fiber 
sensitivity profile. 

We are limited in our ability to compare our data on 
NeuN-stained cells surrounding the implant with prior 
literature, due to the high number and distribution of the 
splaying microfibers. NeuN staining is often evaluated around a 
single implant point, and by measuring the labeled cell density 
in a 50~\si{\micro\meter} region surrounding the implant; in a 
bundle of our splaying optical fibers, that region often 
contains additional optical fibers. Even with this limitation, 
we observe NeuN-stained cell densities in the surrounding region 
that exceed those described for a range of larger, less flexible 
probes and implants 
\cite{Biran:2005dm,Winslow:2010bn,Welkenhuysen:2011cu,Harris:2011dy}. 
These findings suggest the splaying microfibers may provide a less 
invasive option for interfacing with deep brain regions.

Our modeling results define the sensitivity profile of the 
fibers, indicating that fibers optically interface with a 
small volume of tissue near the tip of the fiber. By 
superimposing the per fiber sensitivity profile in a geometry 
consistent with our histology, our model allows evaluating 
the fiber bundle in terms of light delivery (relevant for 
stimulation) and round-trip fluorescence (relevant to recording). 
Although in vivo experimental performance will vary due to tissue 
autofluorescence, and due to indicator brightness and density, 
our model provides intuition for the likely interface properties. 
When used for stimulation, our Monte Carlo simulation suggests 
that patterned illumination of a handful of fibers should 
precisely activate distinct subsets of the neural population in 
the target brain region with minimal crosstalk. When used for 
recording, the round-trip fluorescence will be dominated by neurons 
in the 40~\si{\micro\meter} region surrounding the fiber aperture; collectively, 
a bundle should act as a high channel count fluorescence photometry 
interface capable of sampling fluorescent indicator activity from 
hundreds or thousands of points throughout the target deep brain 
region.

This potential for recording via the fiber bundle is prototyped by 
the measurements of fluorescent microbeads diffusing in solution, 
suggesting that it is possible to excite and measure fluorescence 
through the 5~\si{\micro\meter} cores. The traces from 
neighboring fibers show uncorrelated activity, consistent with 
the splaying of the fibers and minimal crosstalk between fibers. 
Further work is needed to demonstrate the in vivo utility of the 
optical microfibers for optogenetic stimulation and fluorescence 
imaging.

Collectively, these findings indicate that bundles of splaying 
optical microfibers may provide an alternative to GRIN lenses for 
bidirectionally interfacing with deep brain regions. Specifically, 
this technique may achieve a unique compromise in the set of 
tradeoffs associated with extending optical techniques to less 
superficial brain regions. The method provides a high channel count 
interface distributed throughout a non-superficial 3D volume with 
potentially reduced tissue damage relative to GRIN lenses. Additional 
histology in a wider range of species will be needed to compare 
tissue response between implanted fiber bundles and GRIN lenses.

For both recording and stimulating, the self-splaying property means 
that the fiber distribution will be incoherent---the position of the 
fibers is unknown once implanted. Because of this, the fibers are not 
able to elucidate absolute spatial patterns in neural activity. 
Despite not being able to identify the absolute position of the fibers 
in the tissue during usage, selectively illuminating individual fibers 
(e.g., by scanning a laser across the imaging surface with a 
galvanometer) and detecting the amount of light collected by all other 
fibers might serve as a proxy for measuring relative distances between 
fibers in the tissue \cite{Heshmat:2016iq}.

When recording via the bundle, it is only possible to measure a single 
optical intensity value for each fiber, and each fiber will act as the 
optical equivalent of a local field potential. Yet, by examining 
correlations across fibers resulting from overlapping light fields, the 
recorded signals may be amenable to known source separation techniques 
such as independent component analysis \cite{Hyvarinen:2000vk} or 
bayesian source separation \cite{Knuth:2002vo}, achieving an optical 
form of ``spike sorting.''

For stimulation, a digital micromirror device (DMD) can be 
used to project patterned light onto the polished imaging 
surface at the end of the fiber bundle. Given attenuation 
measurements and sensitivity profile modeling, 
2.5~\si{\micro\watt} optical power could be coupled into a 
fiber to stimulate neurons localized near the fiber aperture.

The simulations presented here are limited to normalized, 
noise-free measurements of the optical profile for individual 
fibers and assume uniform and consistent optical probe 
expression within a target neural population modeled as point 
sources. Such simulations enable estimating the the region with 
which the fibers interface, and the relative intensity of 
excitation/stimulation light delivery and the relative round-trip 
fluorescence yield. By incorporating the expression and 
efficiency of the relevant fluorescent probe, the 
model is capable of calculating absolute power measurements and evaluating 
signal to noise performance.

The histology presented here suggests a new class of brain 
implants based on self-splaying microfibers, similar to 
previous work with nickel chromium aluminum 
microelectrode brushes used for chronic electrophysiology 
in primates \cite{Kruger:2010jz}, but using 10--100 times 
more fibers per bundle, where each fiber is approximately 
6 times less stiff (based on the area 
moment of inertia). A large number of ultrasmall fibers can 
be implanted in the brain while minimizing damage near the 
active end of the implant. This principle can apply to optical 
fibers as illustrated here, or to new electrode arrays, such 
as the carbon fiber ultramicroelectrode array 
\cite{Guitchounts:2013bs}, or silicon carbide 
ultramicroelectrodes \cite{Deku:2017el,Pancrazio:2017fi}. As 
high density interconnect solutions are developed for 
ultramicroelectrode arrays, we anticipate seeing the principle 
of self-splaying microfiber interfaces extended to the 
electrical domain \cite{oro52681}.

Self-splaying optical microfibers compliment a number of new 
techniques aiming to achieve high channel count optical 
interfaces, such as multi-site fiber photometry 
\cite{Guo:2015gu} and optoelectronic probes with embedded 
waveguides or with on-device, implantable light sources and 
detectors \cite{Warden:2014bx,Wu:2015gk,Segev:2017en}. 
Technique development relevant to each of these methods will 
have broader repercussions, such as on-device $\mu$LED illumination 
and CMOS sensors to achieve fully head-mounted and wireless 
optical interfaces; silicon waveguides and switches for 
lithographic manufacture of optical implants; and optical 
gratings to better localize and multiplex signals 
\cite{Segev:2015ez}. The self-splaying form factor described 
here will benefit from these developments, with future 
iterations potentially moving illumination and sensing optics 
to a wireless, head-mounted device.
