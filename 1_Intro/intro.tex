\chapter{Introduction}
\label{chapter:intro}

\thispagestyle{myheadings}

% TODO: read through this section

Optical techniques have been a pivotal tool in the 
process of observing and manipulating neural activity, 
achieving a stable interface for interacting with cells 
over a large field of view, thanks in large part to the 
genetic probes that enable resolving individual 
neural contributions \cite{Emiliani:2015jl}. 
Novel genetic probes, such as voltage 
indicators, are only broadening the questions that can 
be answered via optical techniques by achieving 
greater temporal resolution and revealing subthreshold 
activity \cite{Gong:2015is}.

Despite these strengths, optical techniques are severely 
constrained by light scattering. In response to the 
limitations of scattering, a number of innovative approaches 
have been developed, with varying tradeoffs in terms of quality 
and feasibility. Multi-photon microscopy techniques enable 
imaging up to 1~mm below the surface, but require animals to 
be head-fixed and hence limit behavior paradigms 
\cite{Horton:2013gxa}. Attempts to adapt multi-photon microscopy 
to freely behaving animals have achieved neither the 
stability nor the ease of implementation desired 
\cite{Helmchen:2001tw,Flusberg:2005tq}.

Alternatively, more superficial brain regions can either 
be removed \cite{Dombeck:2010jr} or circumvented 
by implanting an optical probe, such as a GRIN lens 
\cite{Barretto:2009hk}, a microprism \cite{Andermann:2013kc}
or a communications-grade optical fiber for fiber 
photometry \cite{Guo:2015gu}. These techniques can be 
effective \cite{Betley:2015cn}, but damage superficial 
tissue and limit imaging to a planar cross-section 
of the desired brain region. 

In this thesis, we propose using bundles of splaying 
microfibers as a new optical interface, which has the 
potential to provide a high channel count, minimally 
invasive, stable, bidirectional optical interface for 
deep brain regions. The thesis is organized as follows: 
first, in \cref{chapter:intro}, we briefly review 
existing optical methods and associated tradeoffs, 
framing the role of the splaying optical microfibers in
the larger methodological landscape. Second, in 
\cref{chapter:histology}, we describe the implant in 
detail and histologically evaluate the implant, 
assessing both the splaying properties of the bundles
and the presence of neurons in close proximity to
the fibers. Next, we describe optical models of both 
the individual fibers and the interface properties of 
the bundles in \cref{chapter:modeling}. In 
\cref{chapter:recording}, we describe initial in vivo
applications of the interface and show fluorescence 
data recorded through the fibers. Finally, in
\cref{chapter:processing}, we describe peripheral 
developments (both hardware and software) that enable 
application of the technique in longitudinal 
experimental work.

\section{Background}

\subsection{Optical techniques}

Due to advantages inherent to optical techniques for 
recording and manipulating neural activity, these techniques 
have become indispensable to advancing systems neuroscience. 
Specifically, optical methods allow sensing and perturbing 
neural activity at a cellular spatial resolution and, due
to new indicators, at a sub-action potential temporal 
resolution. In addition, optical techniques can be deployed 
in long-term experiments to track cellular dynamics over time 
in awake behaving animals, allowing directly probing the 
circuits relevant to complex behaviors.

\paragraph{Probes and indicators.} Crucial to the success of 
optical techniques are the development of genetically encoded 
probes. These proteins enable optically interfacing with 
existing cells. 

One of the earliest and most pervasive probes is GCaMP, a
combination of green fluorescent protein (GFP) isolated in 
jellyfish, and a calcium binding chain. When calcium 
binds to the protein, conformational changes in the chain 
alter the efficiency of the fluorescent protein and, 
as a result, increase its fluorescence
\cite{Nakai:2001fy,Barnett:2017kn}. By illuminating cells expressing such a
probe with the excitation wavelength and measuring the 
fluorescent emissions, it is possible to sense changes in
intracellular calcium and, hence, the calcium-mediated 
depolarization associated with action potentials. Subsequent 
iterations on the protein have increased its efficiency, 
speed and brightness \cite{Chen:2013fc}.

The capacity to measure intracellular calcium enables 
inquiries into a range of signaling and encoding questions, 
but also is inherently limited to suprathreshold cellular 
dynamics. New probes are being developed to sense other 
dimensions of cellular activity, including pH indicators, 
neurotransmitter indicators and voltage indicators 
\cite{Lin:2016id}. Voltage indicators have tremendous 
potential to allow visualizing both subthreshold and 
suprathreshold cellular dynamics at high temporal resolutions 
\cite{Han:2013iz,StPierre:2014db,Gong:2015is}. But given 
the highly localized voltage differentials (only present 
at the cell membrane) and the much shorter time course of 
voltage fluctuations, these probes must be more precisely
localized and much brighter to enable high frame rate 
acquisition and detection.

In addition to sensing other cellular dynamics, new fluorophores  
are being developed and deployed that leverage the  
potential to multiplex signals across wavelengths. For example,
calcium and voltage indicators are being developed that use 
red or near-infrared fluorophores 
\cite{Tischbirek:2015fi,Dana:2016hx}. Such new fluorophores offer 
the potential to multiplex signals across non-overlapping 
wavelengths: this can enable sensing two properties 
simultaneously (e.g., calcium and voltage), or can enable sensing
from two subpopulations simultaneously, or can enable both 
sensing and modulating activity simultaneously. In addition to 
the ability to multiplex signals across wavelengths, new 
fluorophores can extend optical access to deeper brain regions by 
moving to wavelengths with lower absorption 
\cite{Tischbirek:2015fi,Dana:2016hx}.

Beyond sensing, novel genetic probes enable manipulating 
neural activity. Starting with Channelrhodopsin-1 
\cite{Nagel:2002cw}, which encodes a light-gated photon 
channel derived from the phototaxis mechanism in green 
algae, such probes have enabled modulating the excitability 
and membrane potential of neurons through controlled 
exposure to specific excitatory wavelengths 
\cite{Boyden:2005cd,Deisseroth:2006dd,Yizhar:2011jv}.

\paragraph{Genetic targeting.} All of these fluorescent probes 
and light gated channels are genetically encoded, which 
allows for precise targeting and expression of the protein 
in specific neural subpopulations of living animals. 
Transgenic animals are available with expression in cell
types of interest, while new viral vectors enable targeting 
expression in neurons based on connectivity \cite{Tervo:2016go}.
New techniques, such as relying on immediate early genes, 
allow targeting expression to those neurons involved 
in specific circuits and memories \cite{Liu:2012jv}. All 
of these advances in genetic targeting enable precisely
interfacing with a narrowly defined neural population, 
allowing researchers to interrogate the dynamics relevant 
to a specific behavior or circuit.

\paragraph{Field of view.} Optical interrogation and manipulation 
of neural circuits offers a number of additional advantages 
beyond the probes and targeting flexibility. One of the big 
advantages is the ability to interface with neurons over a 
large field of view, providing simultaneous access to thousands 
of neurons \cite{Mohammed:2016fq}. New iterations have scaled 
this up to hemisphere- or skull-sized imaging windows, exposing 
millions of neurons \cite{Kim:2016hh}. Through the large field 
of view, optical techniques provide the ability to track 
information encoding or behavioral modulation throughout a 
region or across regions, providing a more holistic vantage of
the neural activity. Of course, with this breadth of data, new 
analysis challenges emerge, as it becomes more difficult to 
identify the salient activity.

\paragraph{Longitudinal access.} Optical access not only 
achieves a large spatial field of view, it also enables long
term tracking of individual neurons. Cells can be tracked 
across days to understand how neural encoding and activity 
changes with time. For example, recording calcium activity 
in the song bird premotor area exposed instability in 
neural encoding \cite{Liberti:2016bc} that was not visible in 
previous electrophysiology experiments, where single cells 
can often only be tracked on the scale of hours 
\cite{Hahnloser:2002hj}.

\paragraph{Awake behaving experiments.} This goes hand-in-
hand with the longitudinal access, but it is worth 
explicitly emphasizing that optically interfacing with the 
brain does not preclude awake behaving 
experiments. Many interface technologies, discussed in the 
next section, allow recording animals in freely behaving 
paradigms. In the song bird example just mentioned, this 
ability to record neural activity in freely behaving animals 
is crucial, as the birds will rarely sing if constrained 
\cite{LibertiIII:2017df}.

\subsection{Interface technology}

The last section identified 
a range of genetically encoded probes that enable optical 
access to neural activity. A complimentary area of 
technology development has focused on the hardware and 
techniques for recording or stimulating the described 
probes. There are a number of techniques used to precisely 
interface with the neurons in question, each of which 
achieves different tradeoffs in terms of sensitivity, 
signal localization and experimental constraints. We 
highlight three of the most widely used technologies 
below.

\paragraph{Multi-photon microscopy.} Multi-photon microscopy
provides powerful and precise access to fluorescent indicators,
achieving greater precision in the depth axis through the 
two photo absorption and improving penetration by exciting  
with higher wavelength light \cite{Xu:1996ul}. In addition,
the two-photon absorption decreases the amount of excitation light 
hitting the tissue, reducing the chance of bleaching \cite{Denk:1990ws}.
Three-photon microscopy further builds on these principles, 
achieving greater penetration, allowing imaging indicators 
at depths exceeding 1~mm \cite{Horton:2013gxa,Wang:2017jp}.

Despite these strengths, multi-photon microscopy faces key 
constraints. Given the cost associated with constructing and 
maintaining a two-photon microscope, there are substantial fixed 
costs and few labs have the capacity to run simultaneous 
experiments. In addition, the types of experiments are constrained,
often relying on recording in animals that are head fixed. 
Head-mounted two photon microscopes show 
potential \cite{Helmchen:2001tw,Flusberg:2005tq}, but these 
have not achieved the reliability or stability necessary for 
widespread use.

\paragraph{Single photon microscopy and miniature microscopes.}
Singe photon microscopy requires more excitation---risking more
photobleaching---and does not have the depth penetration or 
resolution of multi-photon microscopy, but it carries a number 
of important advantages. Most notably, single photon microscopy 
can readily be scaled to small applications, allowing recording 
from freely behaving animals through miniature head-mounted 
microscopes \cite{Ghosh:2011ee,Cai:2016hm,LibertiIII:2017df}.
These setups come with a lower price tag and enable 
recording animals during less constrained behaviors.

Variations on the standard miniature microscopes are being 
developed that incorporate some of the advantages of multi-photon
microscopy, such as new head-mounted light-field microscopes 
capable of high speed volumetric imaging \cite{Skocek:2018hs}.

\paragraph{Fiber photometry.} Another technology used 
for optically interfacing with the brain is fiber photometry. 
The technique relies on using a large diameter 
(125~\si{\micro\meter} or more), multi-mode fiber to 
deliver excitation light and collect fluorescence from a 
specific region 
\cite{Adelsberger:2005dy,Cui:2013dq,Adelsberger:2014jd}.
Such an approach sacrifices the spatial resolution of 
microscopy for high sensitivity, high temporal resolution 
bulk recording. For example, it enables recording bulk 
fluorescence from axonal projections \cite{Gunaydin:2014dh}. 
Such fiber setups are also frequently used to deliver 
excitation light to opsins in a target region, in order
to modulate neural activity \cite{Warden:2014bx}.

Multi-site fiber photometry has scaled these techniques up to 
8--12 fibers, each of which can be implanted in a different site.
Collectively, the fibers enable simultaneous recording or bulk 
fluorescence across brain regions \cite{Guo:2015gu}.

\paragraph{Waveguides and new implant.} The three methods 
above have widespread usage, but there are a number of new 
approaches being development that combine optical and electrical 
components to move both illumination and detection to an implantable probe.
With such devices, it is possible to use less light power, 
more easily image large volumes and use waveguides to achieve 
targeted excitation \cite{Warden:2014bx,Wu:2015gk,Segev:2017en}.

\subsection{Accessing deep brain regions}

The range of optical techniques described in the previous 
section are widely used to optically record or manipulate 
neural activity. Yet all of these techniques are constrained 
by the light scattering and absorption of the brain, which 
either limits the techniques to more superficial regions or 
requires more invasive implants to access deep brain regions.

\paragraph{Multi-photon microscopy.} Two- and three-photon 
microscopy achieve greater penetration through use of 
wavelengths that have lower scattering coefficients in 
tissue \cite{Wang:2017jp}. This is especially true for 
three-photon microscopy, that is able to record 
fluorescent indicator activity from depths exceeding 1~mm.
But as stated before, these methods are generally 
limited to head-fixed experimental paradigms.

\paragraph{Implantable optics.} To reach further than 
is accessible with three-photon microscopy, or to image 
from a deep brain region in a freely behaving animal, 
researchers currently rely on implantable optics. 
Specifically, GRIN lenses \cite{Barretto:2009hk} and 
prisms \cite{Andermann:2013kc} are frequently used to 
access deeper brain structures.

But as we have learned from an existing body of literature related to 
electrode development, implants with a cross section 
greater than 50~\si{\micro\meter} can cause neuronal 
damage and death over a region up to 100~\si{\micro\meter}
from the implant \cite{Seymour:2007dj}. The insertion 
trauma and the motion of the brain relative to the implant 
after insertion can trigger tissue encapsulation, 
disruption of oxygenation and  excitotoxic cell death
\cite{Szarowski:2003cz,Polikov:2005cq,McConnell:2009hr,Freire:2011gl}.
Implants that have a small cross section ($<$10~\si{\micro\meter}) 
and that are more flexible can avoid this tissue response 
\cite{JohnPSeymour:2006td,Harris:2011dy,Kozai:2012bp,Patel:2018cr}.
Unfortunately, most optical implants have diameters exceeding 
500~\si{\micro\meter} (in order to achieve a wide field of view) 
and are rigid.

As a result, GRIN lenses and other optical implants damage 
or destroy the tissue in the immediate path and 
are often encapsulated in glia due to the induced  
tissue response \cite{Lee:2016ko}. The interface will 
collect fluorescence activity from neurons just beyond this 
region of encapsulation. Yet the full impact of the implant 
is often unknown. Due to dense local connectivity, tissue 
damaged by the implant and foreign body response can impact 
network dynamics in the imaging plane 
\cite{Hayn:2015ew,Hayn:2017kj,GossVarley:2017kf}.

\paragraph{Tissue removal.} Another approach worth mentioning, 
but sharing many of the limitations of optical implants, is 
removal of superficial tissue \cite{Dombeck:2010jr}. By 
removing superficial tissue, it is possible to directly access 
the region of interest and apply standard imaging techniques 
(such as a head-mounted microscope or multi-photon imaging). 
But the contribution of superficial regions and local connectivity 
are both jeopardized or obliterated in such an approach.

\section{Proposed Solution}
%		Bidirectional optical access
%		High channel count
%		Longitudinal access
%		Awake behaving access

In order to translate the benefits inherent in superficial 
imaging techniques, such as long-term stability, high-channel 
counts and minimal tissue damage, to deep brain regions, we 
propose the use of arrays of dissociated optical microfibers 
that can be implanted in brain tissue. The principle of these 
optical fibers is based on self-splaying carbon fiber microthread 
arrays developed for electrophysiology 
\cite{Guitchounts:2013bs,Markowitz:2015ko}. The optical fiber 
bundles contain hundreds or thousands of multimode 
optical microfibers as small as 6.8~\si{\micro\meter} in diameter, 
displacing significantly less brain tissue as 
compared with existing optical fiber implants. As each 
fiber travels independently and finds a path of least 
resistance, the fibers separate and splay, expanding 
the potential recording area while minimizing tissue 
damage. Each fiber maintains near total internal reflection, 
allowing the fiber to optically interface with genetically 
encoded indicators and probes in the tissue surrounding the 
fiber aperture.

The fibers are constructed by sourcing commercial leeched 
fiber bundles produced as flexible endoscopes, which are 
composed of thousands of 8~\si{\micro\meter} glass microfibers, 
where each fiber has a low index of refraction cladding for 
light confinement. By cutting the fiber bundle in half, we 
gain access to the individual fibers, which can be directly 
implanted into the tissue. Outside of the brain, the fibers 
converge to a polished imaging surface where the fibers are 
arranged in a tight lattice that can be mounted under a 
traditional fluorescence microscope objective.

This approach achieves a number of the usual advantages 
of optically interfacing with tissue, while foregoing the 
substantial tissue impact associated with large, monolithic 
implants such as GRIN lenses. By implanting hundreds or 
thousands of optical microfibers, it is possible to record 
or stimulate neurons over a large region of tissue. Given 
that the fibers splay during insertion, a bundle can spread 
out over 1~mm at a depth of 3~mm, sampling fluorescent 
activity over a large 3D volume. Because 
each fiber is flexible and has a small cross section, the 
individual fibers will not have the same impact as a larger,
more rigid implant, preserving more of the local connectivity 
and network dynamics. Finally, because each fiber has 
near total internal reflection, it will deliver excitation 
light to and collect fluorescence from a small amount of 
tissue near the fiber aperture.

Bundles of optical microfibers have the potential to 
extend multichannel, bidirectional optical techniques to 
deeper brain regions without having to remove or obliterate 
the adjacent brain regions. In the next chapter, we 
present a more detailed description of the fibers and 
review histological evidence showing the distribution 
of fibers in tissue and the presence of neurons in close 
proximity to the fiber tips.
