\chapter{Conclusions}
\label{chapter:Conclusions}

\thispagestyle{myheadings}

% path to figures for this chapter
% must have a trailing slash
\graphicspath{{7_Conclusion/Figures/}}

\section{Splaying optical microfibers as an interface}

% TODO: write conclusions

\section{Future work}

This work has presented a new implant and technique
for deep brain imaging, along with a basic survey of 
relevant histology, modeling and initial recordings. 
But these are early steps, and future work will 
continue to adapt this technique to greater potential 
and new use cases.

\subsection{Additional in vivo recordings}

In \cref{chapter:recording}, we presented initial 
in vivo recordings showing fluorescence activity 
from intravenously injected fluorescent agents and 
from genetically encoded calcium indicators during 
both a cortical spreading depolarization and a 
forepaw stimulation experiment. The forepaw results 
are promising, but also highlight challenges in 
terms of contrast-to-noise and moving to chronic 
implants with awake behaving mice.

More work is needed to adapt the chronic recording 
interface described in \cref{chapter:processing} to 
new animal models, including mice. The dexterity of  
mice and their tendency to chew on objects necessitates 
a more active protection of the fiber implant.

Such an interface will allow collecting additional 
sensory-evoked fluorescent activity, without the 
potential confounds of anesthesia. Being able to 
repeat recordings in chronically implanted animals 
will allow the brain to recover post-implant and will 
provide an opportunity to testing design iterations 
to understand how they impact signal quality.

With further refinement, the technique can eventually 
be applied to fundamental systems neuroscience 
questions in deep brain regions.

\subsection{Stimulation and structured illumination}

The splaying optical microfibers have the ability to 
both deliver excitation light to the tissue, and to 
collected fluorescent emissions from neurons 
surrounding the fiber aperture. In concert, this 
enables fluorescent recording, which has been the 
primary focus of this effort. But given that the 
fibers can deliver excitation light, they also 
provide a compelling interface for manipulating 
neural activity using opsins or other light-gated 
ion channels. 

Collectively, illumination of the full bundle can 
modulate neural activity across a large region. But 
the real potential of the bundles as a stimulation 
interface comes from selectively illuminating a 
subset of the fibers, enabling precise activation 
or silencing of a small number of neurons (see 
the modeling in \sref{sec:histology-modeling}). 
Specifically, coupling 2.5~\si{\micro\watt} of 
470~nm into a fiber should provide sufficient 
optical power to activate channelrhodopsin over a 
18,000~\si{\micro\meter}$^3$ region surrounding 
the tip of the fiber.

This becomes especially powerful when combined 
with fluorescent imaging, using a fluorescent 
indicator with orthogonal wavelengths. Imaging 
during a behavior enables identifying those 
fibers which pick up motor or sensory locked 
activity; then selective silencing of those 
neurons via stimulation can help disambiguate 
their contribution to the behavior. The 
multimode fibers enable a bidirectional 
interface, allowing simultaneous fluorescence 
imaging and stimulation.

For stimulation, a similar configuration can be 
used to the chronic recording interface described 
in \cref{chapter:processing}. But rather than 
focusing excitation light across the full fiber  
bundle, a digital micromirror device (DMD) can be 
used to project patterned light onto the polished 
imaging surface at the end of the fiber bundle. 
If mounted on a commutator, real time registration 
or a rotation encoder can enable adapting the 
stimulation pattern to the rotational position of 
the fiber.

\subsection{Multiplexing}

In addition to enabling patterned stimulation, 
structured illumination may have the potential 
to enable better source separability during 
recording by iterating through illumination 
patterns. Inspired by the increased resolution 
achievable using structured illumination 
microscopy
\cite{Gustafsson:2000wj,Gustafsson:2005hg,Saxena:2015fm},
we believe that there may be potential 
to ``multiplex'' activity by rotating through 
different patterns that illuminate distinct 
permutations of the fiber bundle. 

Each illumination pattern will correspond with a 
distinct mixing matrix, based on the differences 
in excitation power. Applying the same source 
separation techniques described in 
\cref{chapter:modeling}, it should be possible 
to estimate each of the mixing matrices and, as 
a result, separate out more of the underlying 
neurons contributing to the fluorescence.

Under such a multiplexed recording paradigm, there 
will be a tradeoff between decreased overall 
excitation (due to illuminating only a subset of the 
fibers) and less temporal resolution (as some 
neural activity may be lost in the time before 
returning to a specific illumination pattern. 
As a result, such recording may benefit from using 
modeling to identify an optimal tradeoff between 
number of illumination patterns and excitation 
power, and from using a slower time course 
fluorescent indicator (e.g., GCaMP6s).

\subsection{Head mounted optics}

As with the move toward head-mounted miniature 
microscopes to record fluorescent activity 
\cite{Ghosh:2011ee,Cai:2016hm,LibertiIII:2017df}, 
the bundles of splaying optical microfibers may 
be most useful once the recording infrastructure 
is sufficiently adapted to a head-mounted 
recording configuration. At least initially, this 
would most likely apply to recording or bulk 
stimulation, which have similar requirements to 
existing miniature microscopes. 

The design of a miniature fluorescence microscope 
closely mirrors the larger fluorescent microscopes 
we used to image through the fiber bundles 
(\fref{fig:miniscope}). An LED light source 
provides excitation light through an excitation 
filter and dichroic, while fluorescent activity 
returns, passing through the dichroic and an 
emission filter to a CMOS. Existing miniature 
microscopes benefit from open and adaptable 
adaptable designs \cite{Cai:2016hm,LibertiIII:2017df},
and with potential modifications to the optical 
path to couple light into the fiber bundle, 
may provide a small, low cost option for 
chronically imaging fluorescent activity in awake
behaving animals.

\begin{figure}
\includegraphics[width=\textwidth]{fig1-miniscope.eps}
\caption[Miniature microscope]{\textbf{Miniature, head-mounted 
microscope with potential to record from fiber bundle.} Existing 
head-mounted miniature microscopes have a compact design modeled 
off traditional single photon fluorescence microscopy. By 
adapting the objective optics, such a configuration could 
provide a stable, chronic interface for imaging through a bundle 
of optical microfibers. \cite{LibertiIII:2017df}}
\label{fig:miniscope}
\end{figure}

The biggest challenge to building a head 
mounted microscope for imaging via the fiber 
bundles is ensuring sufficient sensitivity 
to measure fluorescent signals and, ideally,
to be processed by the described source 
separation techniques. As shown in 
\cref{chapter:recording}, existing 
sensory-evoked recordings have a low 
CBR, necessitating high sensitivity sensors 
and low noise recording environments. With 
further work, it may be possible to improve 
the CBR and may move into a regime where the 
fibers are compatible with the optics and 
sensors available for headed mounted microscopy.

If achieved, a head mounted microscope interface 
for recording from the fibers would offer a number 
of advantages and overcome some of the chronic 
recording challenges described in 
\cref{chapter:processing}. Specifically, an animal 
could be implanted such that only the fiber ferrule 
was exposed above the skull. The miniature 
microscope could either be permanently or temporarily 
fixed above the ferrule and polished imaging surface; 
if temporarily fixed, the fiber ferrule should likely 
be secured under a protective cap when the microscope 
is not in place to protected against accidental 
damage. Such a configuration would eliminate the 
vulnerable fibers and the risk of physical stress on 
the bundle. The shorter fibers will have less 
attenuation, and likely less degradation in attenuation 
over time. And given that the optics and microscope 
are fixed to the head, the optical commutator and 
need to correct rotational motion would both be 
eliminated (an electric commutator or wireless miniature 
microscope would still be needed). Overall, a head-mounted 
recording configuration has tremendous potential for 
longitudinally interfacing with awake behaving animals, 
without the risk of the animal getting tangled or 
damaging the fibers.

