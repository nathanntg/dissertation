\chapter{In vivo recordings show calcium dynamics recorded via splaying optical microfibers}

\label{chapter:recording}
\thispagestyle{myheadings}

% path to figures for this chapter
% must have a trailing slash
\graphicspath{{5_Recording/Figures/}}

\section{Introduction}

Optical techniques for recording fluorescent probes are a cornerstone 
of rapidly advancing systems neuroscience. With an expanding suite of 
genetically encoded probes and techniques for targeting expression, it
 is possible to optically record calcium, voltage and neurotransmitter
 dynamics in a precisely selected neuronal subpopulation 
\cite{Emiliani:2015jl,Gong:2015is}. Yet these techniques are 
inherently constrained by the scattering and absorption properties of 
tissue, constraining optical techniques to more superficial brain
regions.



Three-photon imaging, which is at the cutting edge of optical techniques,
 can record from intact brain tissue at depths exceeding 
1~\si{\milli\meter} \cite{Horton:2013gxa,Wang:2017jp}. But to reach 
regions beyond this depth, current research relies on either removing 
superficial tissue \cite{Dombeck:2010jr} or implanting prisms or 
lenses \cite{Jung:2004kv,Barretto:2009hk,Andermann:2013kc,Cui:2013dq}.
 Such invasive implants can compromise or obliterate adjacent regions 
and potentially jeopardize local connectivity relevant to the neural 
population being investigated.



In an effort to extend optical techniques to deep brain regions that 
both preserves the ability to interface with hundreds of 
neurons---and, as a result, differentiate neural coding throughout the 
region---and that lessens the impact on local network dynamics and 
neighboring brain regions, we proposed a new optical interface in 
\sref{sec:methods-fibers}. Specifically, we constructed implants from 
bundles of hundreds or thousands of small diameter optical microfibers
 (as small as 6.8\si{\micro\meter}). When implanted, each fiber moves 
independently and follows a path of least resistance, causing the 
fibers to splay as they are inserted. At a depth of 
2.9~\si{\milli\meter}, a bundle of 1,000 fibers can spread over 
1~\si{\milli\meter} in diameter. Because of the small diameter 
and flexible fibers, the fibers displace less tissue and elicit less 
tissue response, while still enabling interfacing with a large 3D 
volume of tissue.



Each fiber has a core and a cladding, achieving near total internal 
reflection, allowing the fiber to interface with a small region of 
tissue below the fiber aperture. Outside of the brain, the fibers come
 together in a polished imaging surface held in a metal ferrule, which
 can be fixed below the objective of a traditional fluorescent 
microscope.



In \cref{chapter:histology}, we have described histology evaluating 
the splay of the 
fiber bundles and the presence of NeuN-stained cells in close 
proximity to the fiber tips; in 
\cref{chapter:modeling}, we have modeled 
the interface capabilities and properties of the bundles. 
In this chapter, we share initial in vivo recordings of fluorescence, 
including intravenous fluorescent agents, calcium-linked fluorescence 
during cortical spreading depolarization (CSD) and sensory evoked 
fluorescence resulting from forepaw stimulation. The intravenous 
fluorescent agent and CSD elicit bright, discernible increases in 
fluorescence. We find weak, but significant fluorescent activity 
associated with forepaw stimulation when using larger diameter 
fibers, and identify potential approaches to translate these successes 
to the small diameter fibers.

\section{Methods}

\subsection{Fibers}

The fiber implants are constructed from two commercially available 
sources, resulting in two variations with distinct fiber diameters. 
Bundles of 8~\si{\micro\meter} fibers are made from flexible medical 
endoscopes with 4,500 individual fibers each (Schott 1534180). The 
bundles are made as a coherent imaging bundle with an acid soluble 
glass between the individual fibers, in addition to the standard core 
and cladding; during manufacturing, the acid soluble glass is 
dissolved leaving the dissociated fibers, which increases bundle 
flexibility \cite{Gerstner:2004to}. Each fiber has a core diameter of 
5.1~\si{\micro\meter}, and a numerical aperture of 0.377. The second 
variation, bundles of 37~\si{\micro\meter} fibers, are made as 
incoherent bundles with approximately 350 fibers individual fibers 
(Fiberoptics Technology Incorporated FTIIG24471). Each fiber has a 
core diameter of 34~\si{\micro\meter} and a numerical aperture of 0.6.
 In both cases, the ends of the fibers come together in polished 
imaging surfaces held in ferrules, and the dissociated fibers are 
covered in a flexible silicone or plastic sheathing for protection.



We cut the bundles in half with a razor, allowing access to the 
individual fibers. The protective sheathing is cut back, exposing a 
sufficient length of fibers for the implant. Depending on the desired 
implant size, a fraction of the bundle of fibers is cut away, again 
using a razor. The remaining fibers are secured in a cohesive bundle 
by placing a bead of light-cured acrylic (Flow-It ALC, Pentron 
Clinical) around the individual fibers, approximately 
1.5~\si{\milli\meter} from the tips of the fibers. Once secured, the 
fibers can be further shaped to influence the final tissue geometry; 
we frequently cut fibers with a slight bevel (~100~\si{\micro\meter} 
difference in length) to ensure that fibers sample fluorescence from a
 range of depths in the target region.

\subsection{Surgery}

Animal care and experimental procedures were reviewed and approved by 
the Institutional Animal Care and Use Committee (IACUC) of Boston 
University (protocols 17--017 and 17--026). Recordings were made in 
seven adult mice, including 1 wild type, 3 Thy1-GCaMP3 (Jackson stock 
029860), 2 Thy1-GCaMP6f (Jackson stock 024339) and 1 Thy1-GCaMP6s 
(Jackson stock 024275). Mice were anesthetized with isoflurane during 
surgical procedures (3\% induction, 1--1.5\% maintenance). During 
surgery, the body temperature was monitored via a rectal probe and 
maintained at 37\si{\celsius} by a homeothermic blanket. The animal 
was placed in a stereotaxic frame for the surgery and implant 
procedure. A metal head post was secured to the temporal bone for 
immobilizing the head during recording. Contralateral to the head 
post, a craniotomy was created at either the barrel cortex or the forepaw region 
of the primary somatosensory cortex, sized to match the fiber bundle 
implant (1--1.5~\si{\milli\meter}). The dura within the craniotomy was
 removed.



The fiber bundle was held in a manipulator attached to the stereotaxic
 rig, with the fiber bundle positioned over the durotomy and the 
fibers orthogonal to the surface of the brain. The fibers were slowly 
lowered into the tissue at a rate of approximately 
500~\si{\micro\meter} per minute. After the fibers were lowered to a 
depth corresponding with cortical layer V (500--600~\si{\micro\meter}), 
the bundle was secured to the skull surrounding the craniotomy with 
acrylic.

In three animals, the occipital bone 
was thinned, contralateral to the headpost and ispilateral to the 
fibers, for laser speckle imaging.

In five animals, a frontal craniotomy was made, contralateral to 
the head post, to allow application of KCl in order to induce a 
cortical spreading depolarization (CSD). Until the CSD induction, the 
opening was covered with gel-foam soaked with artificial 
cerebrospinal fluid (ACSF).

\subsection{Recording}

During recording, mice were anesthetized with either isoflurane (1\%) 
or $\alpha$-chloralose (0.5-1\%, IV, 0.1 \si{\milli\liter} per 30 min, 
Sigma-Aldrich C0128). During imaging, the body 
temperature was monitored via a rectal probe and maintained at 
37\si{\celsius} by a homeothermic blanket.


\paragraph{Fiber recording.} The ferrule and polished imaging surface were mounted
 below a traditional fluorescence microscope (Olympus) with either a 
10$\times$ (PL L, 0.25 NA) or a 20$\times$ (Plan, 0.4 NA) objective, 
depending on the imaging surface size. The fibers were illuminated by 
a broadband white LED (Thorlabs SOLIS-3C) at 100\% brightness, with a 
GFP filter cube in the imaging path (Semrock BrightLine GFP-4050B, 
466/40 excitation, 525/50 emission, 495 dichroic). Excitation power at
 the objective was 9~\si{\milli\watt}. Emissions were recorded by a 
sCMOS camera (Hamamatsu ORCA-Flash4.0 v2) with 2$\times$2 binning, 
resulting in a final frame of 1024$\times$1024 16 bit pixels. Exposure
 varied across experiments, but was between 50--100~ms per frame. 
Frames were acquired using HCImage (Hamamatsu) and saved as CXD files.



\paragraph{Intravenous fluorescein isothiocyanate (FITC).} To assess fiber sensitivity to 
fluorescent agents in the vasculature, the fibers were implanted in 
the mice (N = 2) and recorded while injecting fluorescein 
isothiocyanate (FITC)-labeled dextran (molecular weight 10 kDa or 2 
MDa, Sigma-Aldrich FD10S and FD2000S) intravenously 
(50--100~\si{\micro\liter} of 5\% wt/vol in phosphate-buffered 
saline), either via the tail or retro-orbitally \cite{Yardeni:2011fs}.



\paragraph{Inducing cortical spreading depolarization (CSD).} To measure calcium 
fluorescence during a cortical spreading depolarization, a pellet of 
KCl was placed on the surface of the brain of the GCaMP transgenic 
mice (N = 5) at the frontal craniotomy. A drop of ACSF was 
placed on the pellet to dissolve the pellet. This approach is known to
 produce repeated CSDs in anesthetized mice \cite{Karatas:2013ir}. The
 pellet was prepared as described in \cite{Karatas:2013ir}, by placing
 30~mg of KCl in an Eppendorf, adding 10~\si{\micro\liter} of distilled water and 
allowing the water to evaporate.



\paragraph{Laser speckle imaging.} In order to verify the cortical spreading 
depolarization, blood flow was monitored via laser speckle flowmetry. 
The technique is performed as described in 
\cite{Dunn:2001dj,Karatas:2013ir}. A 785~\si{\nano\meter} laser diode 
(Thorlabs L785-P090 controlled by Thorlabs LDC210C) was used to illuminate 
the skull in a diffuse manner. The region was imaged via a 10$\times$ 
objective (M Plan Apo, 0.28 NA) and recorded using a CMOS camera 
(Basler acA2040-90umNIR). A set of 15 raw speckle images was acquired 
at a rate of 90~\si{\hertz}, with 5~ms exposure time and averaged for 
further processing. New sets were acquired at 1-second intervals, and 
were converted to spatial speckle contrast ($K$) using a sliding grid of 
7$\times$7 pixels \cite{Tom:2008hm}. Speckle contrast images were 
converted to correlation time values (using the formula $1/K^2$), which 
are inversely and linearly proportional to mean blood velocity. Relative 
blood flow changes in an ROI ($\sim$0.2$\times$0.2~mm) were calculated 
by dividing correlation time values at each time point by the baseline value 
(calculated by averaging 60 images acquired over 1~min before initiation 
of CSD).

Laser speckle frame acquisition was synchronized with the fiber 
recording by acquiring frame synchronization pulses from both cameras 
over the duration of the recording.



\paragraph{Evoking sensory activity.} To measure calcium fluorescence evoked by 
sensory activity, two wire electrodes were inserted into the forepaw 
contralateral to the fiber implant in the transgenic mice (N = 1). The
 electrodes were connected to a stimulator (Grass Astro-Med S48) via a
 constant current isolation unit (Grass Astro-Med PSIU6), configured 
to output 20~ms, 0.6--1.2~\si{\milli\ampere} square pulses every 2 
seconds. The stimulation was visually confirmed through observing 
associated muscle contractions.



Stimulation pulses were synchronized with the fiber recording by 
acquiring two signals channels, one corresponding with the fiber CMOS 
frame acquisition and one corresponding with the stimulation pulses, 
over the duration of the recording.

\subsection{Analysis}

Fiber recordings were converted from CXD files to raw binary image 
sequences using a custom Python script. Frames were then processed via
 a custom pipeline implemented in MATLAB (Mathworks). During the first
 pass, frames were motion corrected using an implementation of the 
Scale-Invariant Feature Transform (SIFT) algorithm 
\cite{vedaldi08vlfeat,lowe1999object,Lowe:2004kp} and then calculating
 the 1st and 99th percentile pixel intensities across all video frames.
By calculating a pixel range image (by subtracting 
the 1st percentile intensity from the 99th percentile intensity for each pixel), 
fibers with dynamic fluorescent activity are clearly visible. 
MATLAB's circle finding algorithm \cite{Atherton:1999dl} is used to 
automatically draw regions of interest (ROIs) around fibers with high 
variability. To minimize artifacts from motion correction, ROIs are 
shrunk to 75\% of their original size. The analysis pipeline then 
makes a second pass through the video, applying the same motion 
correction transform and averaging the pixel intensity for all pixels 
in each fiber region of interest. The end result is an intensity time 
course for each fiber of interest. Traces are converted to 
$\Delta$F/F$_0$, where F$_0$ corresponds with the median intensity in the 
period prior to the fluorescence onset.

To test the statistical significance of the sensory evoked behavior, a
 Student's t-test was performed comparing the background subtracted 
fluorescence in the 100~ms after the stimulation to the background 
subtracted fluorescence in the 100~ms before the stimulation for those
 fibers where visual inspection identified a potential time locked 
change in fluorescence (N = 2).

\section{Results}

In order to demonstrate and evaluate the recording capability of the 
bundles of optical microfibers, we performed a series of recordings 
across a range of externally evoked fluorescent events.

\paragraph{Intravenous fluorescein (FITC).} Sensitivity to fluorescent agents in 
the vasculature was measured by injecting high molecular weight 
fluorescein conjugated dextran, while recording fluorescence via the 
implanted fibers. \Fref{fig:iv-fitc} shows fluorescent traces for a retro-orbital 
injection of 2 MDa FITC. The fluorescent bolus is visible 1--2 seconds
 after injection, and has more sustained fluorescence than the higher 
molecular weight FITC (not shown). The fluorescent bolus corresponded 
with an average increase in fluorescent intensity of 0.22$\times$ the 
F$_0$ intensity (std. dev. 0.011) in the 25 brightest fibers.

\begin{figure}
\includegraphics[width=\textwidth]{fig2-fitc.eps}
\caption[Recording of FITC in vasculature]{\textbf{Detection of fluorescent agent in vasculature via implanted fibers.} At
 time 0, the mouse received an IV injection of high molecular weight 
FTIC retro-orbitally. Recordings via 8~\si{\micro\meter} fibers show 
an immediate and sustained increase in fluorescence. Each line is the 
$\Delta$F/F$_0$ intensity for a single fiber from the recording.}
\label{fig:iv-fitc}
\end{figure}

\paragraph{Induced cortical spreading depolarization (CSD).} In 
transgenic animals 
expressing a calcium sensitive fluorescent probe, the ability to 
record calcium fluorescence was measured during a cortical spreading 
depolarization by dissolving a pellet of KCl on the surface of the 
brain, which is known to evoke multiple CSDs in anesthetized mice 
\cite{Karatas:2013ir}. We recorded for 20 minutes, beginning at the 
time of application of the KCl. In four of the five animals, we 
observed one or more fluorescent events that corresponded with a CSD. 
Fluorescence, shown in \fref{fig:csd}, increased over 24.1~s $\pm$ 16.8 (std.
 dev), and recovered more slowly, over 87.7~s $\pm$ 17.0 (std. dev). 
This timing is consistent with previous measures of intracellular 
calcium in layer V cells during CSD \cite{Gniel:2010jn}, but with a 
slower rise time consistent with the fact that a fiber will collect 
fluorescence from multiple neurons in its sensitivity profile.

\begin{figure}
\includegraphics[width=\textwidth]{fig3-csd.eps}
\caption[Recording of fluorescence during CSD]{\textbf{Cortical spreading depolarization (CSD) evokes a fluorescence 
increase.} Three recordings from Thy1-GCaMP6f mice implanted with 
fibers after dissolving a pellet of KCl on the surface of the brain. 
Each line is the $\Delta$F/F intensity for a single fiber from the 
recording. Within three minutes of application, a distinct 
fluorescence increase is visible. By recording relative blood flow via
 laser speckle, it is possible to confirm the presence of a CSD. (a) 
Fluorescence associated with a single CSD, recorded via 
8~\si{\micro\meter} fibers. (b) and (c) Synchronized recording of 
laser speckle and fiber fluorescence via 8~\si{\micro\meter} fibers, 
showing a CSD and two subsequent fluorescent events. (d) and (e) 
Synchronized recording of laser speckle and fiber fluorescence via 
37~\si{\micro\meter} fibers, showing two CSDs and an additional 
fluorescent event in between.}
\label{fig:csd}
\end{figure}

As a CSD propagates across the cortex at approximately 
2~\si{\milli\meter} per minute \cite{Ochs:1960jx,Pietrobon:2014gn}, we
 can look at the acquired images of the fiber bundle for timing 
differences in fiber fluorescence that corresponds with this 
propagation. \Fref{fig:csd-frames} shows frames from the acquired video, 
illustrating the propagation of the CSD.

\begin{figure}
\includegraphics[width=\textwidth]{fig4-csd-frames.eps}
\caption[Images of fiber bundle during CSD]{\textbf{Frames of polished imaging surface show progression of cortical 
spreading depolarization (CSD).} Three minutes after dissolving
 a KCl pellet on the brain of a Thy1-GCaMP6f mouse, the fluorescence 
associated with a CSD was visible via the implanted 
8~\si{\micro\meter} fibers. Four background subtracted frames selected
 from the first 33 seconds of the CSD show the spatial propagation of 
the depolarization.}
\label{fig:csd-frames}
\end{figure}

Relative blood flow measured via laser speckle near the fiber implant 
site provided confirmation of the CSD. Because the laser speckle was 
recorded at a second location, there is a time offset associated with 
the CSD propagation. The CSDs identified via blood flow correspond 
closely with recorded fluorescence increases, but we also observe 
additional subsequent fluorescent events that do not have the 
corresponding change in relative blood flow.

For the 8~\si{\micro\meter} fibers, the observed fluorescence increase
 is 0.154$\times$ the F$_0$ intensity (std. dev. 0.03) in the 25 
brightest fibers. For the 37~\si{\micro\meter} fibers, the increase is
 0.577$\times$ the F$_0$ intensity (std. dev. 0.19).

\textbf{Sensory evoked activity.} To record fluorescence evoked by sensory 
stimulation, we recorded from the primary somatosensory region of 
cortex of a transgenic mouse expressing a calcium sensitive 
fluorescent probe while electrically stimulating the contralateral 
forepaw. We align and average fluorescence from 0.4~s before until 
1.6~s after the stimulation pulse. When recording from the 
8~\si{\micro\meter} fibers, we did not observe fluorescent activity 
corresponding with the stimulation pulses. When recording from the 
37~\si{\micro\meter} fibers, we see a small, but significant increase 
in fluorescence over baseline for two fibers (\fref{fig:paw}). The increase in 
fluorescence averages 0.001$\times$ the F$_0$ intensity ($p \leq 0.005$), based 
on 100 aligned stimulation pulses.

\begin{figure}
\includegraphics[width=\textwidth]{fig5-paw.eps}
\caption[Recording of fluorescence during forepaw stimulation]{\textbf{During forepaw stimulation, two 37~\si{\micro\meter} fibers show time 
locked fluorescent activity.} The three traces show stimulation-aligned
 average fluorescent intensity of distinct fibers after background 
subtraction over 100 trials. The fibers are implanted in primary 
somatosensory region of a Thy1-GCaMP6s mouse, and the contralateral 
forepaw receives a 20~ms, 1.2~\si{\milli\ampere} pulse at time 0. The 
shaded region representing the 95\% confidence interval by 
bootstrapping trials. The top two traces have a significant increase 
in fluorescence over pre-trial frames (t-test, $p \leq 0.005$). Bottom, right:
 image of the polished imaging surface with the three corresponding 
fibers highlighted.}
\label{fig:paw}
\end{figure}

\section{Discussion}

These initial recordings demonstrate the potential utility of bundles 
of optical microfibers as a method for recording fluorescence 
activity; each fiber delivers excitation light and samples 
fluorescence from a small region of tissue at the fiber aperture.

First, we demonstrate the ability to sense a fluorescent agent in the 
vasculature by intravenously injecting high molecular weight FITC. 
This initial proof of concept shows the fibers are functional and 
sensitive to a high fluorescence event, and may have promise for 
recording dynamics associated with the vasculature.

Next, we turn our attention to sensing genetically-expressed 
fluorescent probes (in this case, GCaMP6). In these experiments, we 
explore the most intriguing aspect of the splaying optical fiber 
bundles, where the fibers can sample neural activity throughout a 3D 
volume of tissue while retaining intact neurons in close proximity to the 
fibers and, as a result, decreasing the likelihood of a large tissue 
response that may interfere with local connectivity and network 
dynamics (see \sref{sec:results-histology}).

During a bright and distributed event such as a CSD, we see a distinct
 increase in fluorescence that is consistent with previously described
 calcium dynamics during CSD \cite{Gniel:2010jn}. The change in fluorescence 
was bright---clearly visible while recording, before application of 
any preprocessing steps such as background subtraction. In addition to
 seeing the fluorescence corresponding with CSDs identified via the 
laser speckle recording, we see additional, subsequent increases in 
calcium fluorescence that do not have an accompanying change in blood 
flow. This potentially suggests repeated influxes of calcium and/or 
depolarization that may warrant additional investigation.

The CSD recordings also allow a comparison across the two types of 
fibers. When using the larger diameter fibers (37~\si{\micro\meter}), 
the change in fluorescence is almost four times greater than when 
using the smaller diameter fibers (8~\si{\micro\meter}). This is 
consistent with the wider aperture enabling greater light acceptance. 
From the greater sensitivity of the large fibers, we can observe 
additional dynamics in the fluorescence; specifically, after the first
 CSD-evoked fluorescence increase, the fluorescence does not fully 
return to baseline. This may reflect the fact that intracellular 
calcium levels drop quickly after the CSD, but full restoration of 
pre-CSD concentrations takes more than 5 minutes, or may reflect 
variability in the response to the CSD across cortical layers 
\cite{Gniel:2010jn}.

We observed CSDs in four of the five animals following application of 
the KCl. Two of the five animals expressed GCaMP3 (as opposed to 
GCaMP6f) and were excluded from the quantification of brightness given
 the different properties of the fluorescent probe. In one of these 
GCaMP3 mice, we did not observe an increase in fluorescence following 
application of the KCl pellet. We believe that if a CSD occurred, it 
was after our 20 minute recording session. Following the recording 
session, we discovered that the homeothermic blanket malfunctioned and
 the mouse's body temperature dropped by 2\si{\celsius}, which may 
account for the lack of CSD.

Finally, the results recording sensory evoked fluorescent activity are
 more mixed, in part a result of the small sample size (one animal, 
with two recording sessions, one in each hemisphere). With the larger 
diameter fibers (37~\si{\micro\meter}), we did observe statistically 
significant stimulation-locked fluorescence, but the $\Delta$F/F 
change was orders of magnitude smaller than the CSD recording through 
the same fibers. For the smaller diameter fibers 
(8~\si{\micro\meter}), no identifiable stimulation-locked fluorescence
 was observed.

A number of factors likely contribute to the limited or absent 
fluorescent signal, and future work can improve the signal. All 
recordings were taken under isoflurane anesthesia, which suppresses 
sensory evoked cortical activity \cite{Sitdikova:2013fn}. During the 
recordings, we attempted to transition to $\alpha$-chloralose, but were 
unable to sustain the animal's anesthetized state; as a result, we 
continued light isoflurane, potentially interfering with the cortical 
response to the forepaw stimulation. During a subsequent round of 
recording via the larger diameter fibers, we found no stimulation-locked 
fluorescence, consistent with an observed deepening in the animal's 
anesthesia and corroborating the potential role of anesthesia in 
suppressing sensory evoked signals.

A second challenge is that all recordings reported here are acute. 
Acute recordings may be detrimentally impacted by bleeding, swelling 
or other acute reactions to the implant that have not yet been 
evaluated. Consistent with previously reported histology 
(\sref{sec:results-histology}), we believe that the strength of the optical 
microfibers as an interface comes from their chronic stability. Future
 work will focus on building stable chronic implants for mice to 
enable recording from awake, behaving animals, eliminating the effects
 of anesthesia and allowing the surrounding tissue to stabilize after 
the implant process.

Third, the fiber implant process is currently guided solely on 
coordinates. Given the dimpling that can occur during implant 
(\sref{sec:methods}), the implant depth would benefit from closed 
loop feedback. By monitoring fluorescence while lowering the fiber 
bundle into the tissue, it may be possible to more precisely target a 
specific cortical layer.

In addition, excitation power recorded through the larger fibers may 
have been adversely affected by a mismatch between the objective NA 
(0.25) and the fiber NA (0.6), resulting in lost excitation signal. 
Correcting this mismatch may increase the coupling of light from the 
fiber into the objective, and hence may improve the recorded signal.

These initial recordings demonstrate the feasibility of bundles of 
hundreds or thousands of microfibers as an interface to optically 
access 3D volumes of tissue. Incorporating the insights from these 
initial recording attempts will enable improving surgical protocols 
and moving towards an awake, behaving paradigm. With these iterations,
 we hope to improve sensitivity to sensory-evoked fluorescence 
activity, such that it can be c recorded through both the 
small and large diameter fibers.


