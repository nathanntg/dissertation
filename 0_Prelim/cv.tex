\addcontentsline{toc}{chapter}{Curriculum Vitae}

\thispagestyle{empty}

\begin{center}
{\LARGE {\bf CURRICULUM VITAE}}\\
\vspace{0.25in}
{\large {\bf L. Nathan Perkins}}
\end{center}

\newenvironment{list1}{
  \begin{list}{\ding{113}}{%
      \setlength{\itemsep}{0in}
      \setlength{\parsep}{0in} \setlength{\parskip}{0in}
      \setlength{\topsep}{0in} \setlength{\partopsep}{0in} 
      \setlength{\leftmargin}{0in}}}{\end{list}}

\newenvironment{list2}{
  \begin{list}{$\bullet$}{%
      \setlength{\itemsep}{0in}
      \setlength{\parsep}{0in} \setlength{\parskip}{0in}
      \setlength{\topsep}{0in} \setlength{\partopsep}{0in} 
      \setlength{\leftmargin}{0.25in}}}{\end{list}}

% HEADER
%\name{\Large L. Nathan Perkins}
%\address{22 Sacramento Place, Cambridge, MA 02138 $\bullet$ nathan@maxmo.net $\bullet$ (505) 750 - 7757}

% ACADEMICS
\section*{Academics}

\textbf{Ph.D., Computational Neuroscience}, Boston University \hfill \textbf{2014-2018} \\
	Course work included: neuroscience; image and video processing; statistics; neural modeling

\bigskip

\noindent
\textbf{M.S., Technology \& Policy}, Massachusetts Institute of Technology \hfill \textbf{2009-2011} \\
	Coursework included: technology and policy core; applied probability; artificial intelligence; organization measurement; lean business practices

\bigskip

\noindent
\textbf{B.A., Neuroscience}, University of Southern California \hfill \textbf{2003-2007} \\
	Minors: Law \& Society; Thematic Approaches to Humanities \& Society
	
\section*{Research}

\textbf{Bio Optical \& Acoustic Spectroscopy Lab}, Boston University \hfill \textbf{2017-2018} \\
	Advisor: David A. Boas \\
	Research assistant developing minimally invasive, bidirectional, multichannel interfaces for deep brain regions

\bigskip

\noindent
\textbf{Neural Circuit Formation Lab}, Boston University \hfill \textbf{2015-2018} \\
	Advisor: Timothy J. Gardner \\
	Research assistant developing minimally invasive, bidirectional, multichannel interfaces for deep brain regions

\bigskip

\noindent
\textbf{Lean Advancement Initiative}, Massachusetts Institute of Technology \hfill \textbf{2009-2011} \\
	Advisors: Ricardo Valerdi \& Deborah Nightingale \\
	Research assistant focusing on measuring organizational structure and supply chain dynamics.

% PUBLICATIONS
\section*{Publications}

\textbf{Perkins, L. N.}, Devor, A., Gardner, T. J., Boas, D. A. ``Extracting individual neural activity recorded through splayed optical microfibers.'' \textit{In Preparation}

\bigskip

\noindent
Liberti, W. A., Shen, J., \textbf{Perkins, L. N.}, Gardner, T. J. ``Context dependent variability of HVC projection neurons during practice and performance of song.'' \textit{In Preparation}

\bigskip

\noindent
Martinet, L.-E.*, Kramer, M. A.*, Viles, W., \textbf{Perkins, L. N.}, Spencer, E., Chu, C. J., Cash, S. S., Kolaczyk, E. D. ``Robust dynamic community tracking with applications to human brain functional networks.'' \textit{In Preparation}

\bigskip

\noindent
\textbf{Perkins, L. N.}, Semu, D., Shen, J., Boas, D. A., Gardner, T. J. ``High-density microfibers as a potential optical interface to reach deep brain regions.'' \textit{In Review}

\bigskip

\noindent
Pearre, B., \textbf{Perkins, L. N.}, Markowitz, J. E., Gardner, T. J. ``A fast and accurate zebra finch syllable detector.'' PLoS ONE, volume  12, issue 7 (2017): e0181992.

\bigskip

\noindent
Liberti, W. A.*, \textbf{Perkins, L. N.*}, Leman, D. P., Gardner, T. J. ``An open source, wireless capable miniature microscope system.'' Journal of Neural Engineering, volume 14, issue 4 (2017): 045001-10.

\bigskip

\noindent
Liberti, W. A.*, Markowitz, J. E.*, \textbf{Perkins, L. N.}, Leman, D.P., Liberti, D. C., Guitchounts, G., Velho, T., Lois, C., Kotton, D. N., Gardner, T. J. ``Unstable neurons underlie a stable learned behavior.'' Nature Neuroscience, volume 19, issue 12 (2016): 1665-1671.

\bigskip

\noindent
\textbf{Perkins, L. N.} ``Convolutional neural networks as feature generators for near-duplicate video detection.'' BU Department of Electrical and Computer Engineering, Technical Report ECE-2015-05 (2015).

\bigskip

\noindent
\textbf{Perkins, L. N.*}, Marshall, T.* ``Color outlier detection for search and rescue.'' BU Department of Electrical and Computer Engineering, Technical Report ECE-2015-01 (2015).

\bigskip

\noindent
\textbf{Perkins, L. N.} ``An abridged enterprise assessment model to promote consistent reassessment: model development, assessment process and results analysis.'' MIT Thesis (2011).

\bigskip

\noindent
\textbf{Perkins, L. N.}, Abdimomunova, L., Valerdi, R., Shields, T., Nightingale, D. ``Insights from enterprise assessment: How to analyze LESAT results for enterprise transformation.'' in the journal of Information Knowledge Systems Management (IKSM), volume 9, issue 3-4 (2010): 153-174.

\bigskip

\noindent
* indicates co-authorship

% PRESENTATIONS
\section*{Presentations}

``Optical interface for deep brain recording and stimulation through minimally invasive, splaying microfibers.'' BRAIN Initiative Investigators Meeting, Bethesda (2018).

\bigskip

\noindent
``Extending optical techniques to deep brain regions through minimally invasive, splaying optical microfibers.'' Neuroscience, Washington (2017).

\bigskip

\noindent
``Counting the love: Collecting U.S. protest data and analyzing trends in civic discourse.'' Plenary session at Personal Democracy Forum, New York (2017).

\bigskip

\noindent
``High-density microfiber interfaces for deep brain optical recording and stimulation.'' Neuroscience, San Diego (2016).

\bigskip

\noindent
``Developing systems thinking competencies through facilitated simulation experiences.'' Conference of Systems Engineering Research, Los Angeles (2011).

\bigskip

\noindent
``Organizational assessment models for enterprise transformation.'' INCOSE International Symposium, Chicago (2010).

\bigskip

\noindent
``Healthcare Reborn: A systems thinking accelerator.'' International Conference on Management, Behavioral Sciences and Economics Issues (IC-MBSE), Fairfax (2010).

\bigskip

\noindent
``Waning individuality in light of emergence.'' Thematic Options Research Conference, Los Angeles (2007).

% HONORS AND AWARDS
\section*{Honors \& Awards}

\textbf{Best thesis nomination}, MIT Technology and Policy Program \hfill \textbf{2011}

\noindent
\textbf{First Place, Stevens Institute Experience Accelerator} \hfill \textbf{2010} \\
	Created Healthcare Reborn with two other students, a game simulating emergency room patient flow

\noindent
\textbf{USC Renaissance Scholar and Award Winner} (one of ten) \hfill \textbf{2007} \\
	Honors students with broad interests and high academic performance

% WORK EXPERIENCE
\section*{Work Experience} 

\textbf{Arctic Reservations, founder and developer} \hfill \textbf{2005-2016}
%	arcticreservations.com
	
	\begin{list2}
		\item Created software for outdoor adventure outfitters, with comprehensive functionality for reservation management, inventory allocation, accounting, customer relations, reporting, photo sharing, credit card processing, guest registration, post-trip evaluations and email automation
%		\item Served 50 clients in 20 states with one employee and annual revenues in excess of \$250,000
	    \item Over 40,000 guests went on trips booked through Arctic Reservations each month
	\end{list2}

\bigskip

\noindent
\textbf{MaxMo Technologies, independent consultant and developer} \hfill \textbf{1999-2014}
	
	\begin{list2}
		\item Developed and maintained a social network for financial traders, including a comment system with over one million comments and heavy daily traffic, along with an accompanying iOS app
%		\item Created multiple charting tools for visualizing financial and market data 
%		\item Developed neural network and adaptive boosting libraries for machine learning
		\item Built a search engine with a custom data storage engine allowing for millisecond response times and linguistic document clustering, implemented at an education-focused nonprofit
%		\item Created a staff scheduling and management platform for television stations
	\end{list2}

% VOLUNTEER WORK
\section*{Volunteer Work} 
	
\textbf{Peace Corps Volunteer, serving in South Africa} \hfill \textbf{2007-2009}
	
    \begin{list2}
	    \item Focused on capacity building and community development, seeking to strengthen local institutions and resources in order to address the continuing HIV/AIDS pandemic affecting the region
	    \item Assisted in reviews of business and income generating projects; developed a food security garden; revised financial and management strategies; helped secure new grants
    \end{list2}
        
%\textbf{Corner Pixel, co-founder} \hfill \textbf{2004-2010}
%    
%    \begin{list2}
%	    \item Aided nonprofits in establishing an online identity (branding) and connecting with the community
%	    \item Worked with the Dennis Chavez Foundation and the National Institute of Flamenco
%    \end{list2}

% OPEN SOURCE
\section*{Open Source}
	\textbf{Warp detector.} Software capable of calculating a rolling spectrogram for incoming audio and using dynamic programming to compare it with a template signal in order to detect specific syllables. Capable of running on computers or realtime embedded systems, the system achieves millisecond latency. This is used in lab to provide precisely triggered auditory feedback to canaries. (2018)

\bigskip

\noindent
	\textbf{Consensus Contours.} C and MEX implementations of the consensus contour algorithm described by Yoonseob Lim, Barbara Shinn-Cunningham and Tim Gardner, able to extract a consensus of stable time-frequency contours from a time domain signal (such as audio). The algorithm is implemented for rolling signals and leverages optimized, vectorized instructions, providing the first implementation capable of realtime processing of audio. (2017)

\bigskip

\noindent
	\textbf{Syllable detector.} Calculates a spectrogram for incoming audio and processes it using a neural networks in order to detect specific ``syllables.'' Uses vector operations to achieve millisecond latency and low CPU load. This is used in lab to provide precisely triggered auditory feedback to zebra finches. (2016)

\bigskip

\noindent
	\textbf{Find audio.} A set of MATLAB tools for matching instances of an auditory template in a longer signal, as well as warping the time of two auditory sequences to common timestamps. Includes a C MEX implementations of the dynamic programming algorithm. (2016)

\bigskip

\noindent
	\textbf{Video capture.} Software to interface with fluorescence microscopes for video and data acquisition. Performs near realtime processing of inputs, including region of interest extraction and audio processing, enabling construction of basic brain machine interfaces. This is used in lab in conjunction with head-mounted miniature microscopes. (2015)

\bigskip

\noindent
	\textbf{Dynamic plex propagation.} The MATLAB implementation of an algorithm that can identify and track communities in functional networks over time under uncertainty. This has been used by researchers at BU and MGH to understand the propagation of seizures based on ECoG data. (2015)

\bigskip

\noindent
	Code and projects on GitHub at \texttt{https://github.com/nathanntg}
	
%	\textbf{Lin-train} and \textbf{AdaBoost}. Open 

% PROGRAMMING
\section*{Programming} 

	\textbf{Languages (all with 5 or more years experience)}: C, C++, Swift (since release), Python, MATLAB, JavaScript, PHP, SQL, CSS, HTML

\bigskip

\noindent
	\textbf{Other technological proficiencies}: Apache, MySQL (running a multi-server cluster with 100GB data and 99.99\% uptime), Memcache, Git version control
