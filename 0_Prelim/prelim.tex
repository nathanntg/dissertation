% This file contains all the necessary setup and commands to create
% the preliminary pages according to the buthesis.sty option.

\title{High-density microfibers as a deep brain bidirectional optical interface}

\author{L. Nathan Perkins}

% Type of document prepared for this degree:
%   1 = Master of Science thesis,
%   2 = Doctor of Philosophy dissertation.
%   3 = Master of Science thesis and Doctor of Philosophy dissertation.
\degree=2

\prevdegrees{B.A., University of Southern California, 2007\\
	M.S., Massachusetts Institute of Technology, 2011}

\department{Graduate Program in Neuroscience}

% Degree year is the year the diploma is expected, and defense year is
% the year the dissertation is written up and defended. Often, these
% will be the same, except for January graduation, when your defense
% will be in the fall of year X, and your graduation will be in
% January of year X+1
\defenseyear{2018}
\degreeyear{2018}

% For each reader, specify appropriate label {First, Second, Third},
% then name, and title. IMPORTANT: The title should be:
%   "Professor of Electrical and Computer Engineering",
% or similar, but it MUST NOT be:
%   Professor, Department of Electrical and Computer Engineering"
% or you will be asked to reprint and get new signatures.
% Warning: If you have more than five readers you are out of luck,
% because it will overflow to a new page. You may try to put part of
% the title in with the name.
\reader{First}{David A. Boas, PhD}{Professor of Biomedical Engineering}
\reader{Second}{Timothy J. Gardner, PhD}{Associate Professor of Biology}
\reader{Third}{Thomas Bifano, PhD}{Professor of Mechanical Engineering} % if chair, can not be reader

% The Major Professor is the same as the first reader, but must be
% specified again for the abstract page. Up to 4 Major Professors
% (advisors) can be defined. 
\numadvisors=1
\majorprof{David A. Boas, PhD}{{Professor of Biomedical Engineering}}
%\majorprofb{Timothy J. Gardner, PhD}{{Associate Professor of Biology}}
%\majorprofc{Ian G. Davison, PhD}{{Assistant Professor of Biology}}
%\majorprofc{Jeffrey P. Gavornik, PhD}{{Assistant Professor of Biology}}
%\majorprofc{Jerome Mertz, PhD}{{Professor of Biomedical Engineering}}
%\majorprofc{Thomas Bifano, PhD}{{Professor of Mechanical Engineering}}
%\majorprofc{Ian G. Davison}{{Assistant Professor of Biology}}
%\majorprofc{Ian G. Davison}{{Assistant Professor of Biology}}
%\majorprofc{Ian G. Davison}{{Assistant Professor of Biology}}

%%%%%%%%%%%%%%%%%%%%%%%%%%%%%%%%%%%%%%%%%%%%%%%%%%%%%%%%%%%%%%%%  

%                       PRELIMINARY PAGES
% According to the BU guide the preliminary pages consist of:
% title, copyright (optional), approval,  acknowledgments (opt.),
% abstract, preface (opt.), Table of contents, List of tables (if
% any), List of illustrations (if any). The \tableofcontents,
% \listoffigures, and \listoftables commands can be used in the
% appropriate places. For other things like preface, do it manually
% with something like \newpage\section*{Preface}.

% This is an additional page to print a boxed-in title, author name and
% degree statement so that they are visible through the opening in BU
% covers used for reports. This makes a nicely bound copy. Uncomment only
% if you are printing a hardcopy for such covers. Leave commented out
% when producing PDF for library submission.
%\buecethesistitleboxpage

% Make the titlepage based on the above information.  If you need
% something special and can't use the standard form, you can specify
% the exact text of the titlepage yourself.  Put it in a titlepage
% environment and leave blank lines where you want vertical space.
% The spaces will be adjusted to fill the entire page.
\maketitle
\cleardoublepage

% The copyright page is blank except for the notice at the bottom. You
% must provide your name in capitals.
\copyrightpage
\cleardoublepage

% Now include the approval page based on the readers information
\approvalpage
\cleardoublepage

% Here goes your favorite quote. This page is optional.
\newpage
%\thispagestyle{empty}
\phantom{ }
\vspace{4in}

\begin{singlespace}
\begin{quote}
  \textit{Homo sapiens is the species that invents symbols in which to invest passion and authority, 
  then forgets that symbols are inventions.} \\
  \phantom{ }\hfill{Joyce Carol Oates}
\end{quote}
\end{singlespace}

% \vspace{0.7in}
%
% \noindent
% [The descent to Avernus is easy; the gate of Pluto stands open night
% and day; but to retrace one's steps and return to the upper air, that
% is the toil, that the difficulty.]

\cleardoublepage

% The acknowledgment page should go here. Use something like
% \newpage\section*{Acknowledgments} followed by your text.
\newpage
\section*{\centerline{Acknowledgments}}
% TODO: write acknowledgements

The research presented here is only possible due to the support and 
contributions of many individuals. I am especially grateful to all my 
colleagues and collaborators at both the Laboratory of Neural 
Circuit Formation and the Bio Optical \& Acoustic Spectroscopy Lab. 
Both laboratories provided a close knit and collaborative environment; 
fellow researchers lent expertise, guidance, insight and moral support 
that helped advance this work. 

Specifically, K{\i}v{\i}lc{\i}m K{\i}l{\i}\c{c},
Dawit Semu, Jun Shen and Blaire Lee



%Here go all your acknowledgments. You know, your advisor, funding agency, lab
%mates, etc., and of course your family.
%
%\vskip 1in
%
%\noindent
%Nathan

\cleardoublepage

% The abstractpage environment sets up everything on the page except
% the text itself.  The title and other header material are put at the
% top of the page, and the supervisors are listed at the bottom.  A
% new page is begun both before and after.  Of course, an abstract may
% be more than one page itself.  If you need more control over the
% format of the page, you can use the abstract environment, which puts
% the word "Abstract" at the beginning and single spaces its text.

\begin{abstractpage}
% ABSTRACT

Optical interrogation and manipulation of neural dynamics 
is a cornerstone of systems neuroscience. Genetic targeting 
enable delivering fluorescent indicators and opsins to 
specific neural subpopulations. Optic probes can 
fluorescently sense and convey calcium, voltage, and 
neurotransmitter dynamics. This optical toolkit enables 
recording and perturbing cellular-resolution activity in 
thousands of neurons across a field of view.

Yet these techniques are limited by the light scattering 
properties of tissues. The cutting edge of microscopy, 
three-photon imaging, can record from intact tissues at 
depths up to 1 mm, but requires head-fixed experimental 
paradigms. To access deeper layers and non-cortical 
structures, researchers rely on optical implants, such as 
GRIN lenses or prisms, or the removal of superficial 
tissue. 

In this thesis, we introduce a novel implant for 
interfacing with deep brain regions constructed from 
bundles of hundreds or thousands of dissociated, small 
diameter ($<$8~\si{\micro\meter}) optical fibers. During 
insertion into the tissue, the fibers move independently, 
splaying through the target region. Each fiber achieves 
near total internal reflection, acting as a bidirectional 
optical interface with a small region of tissue near the 
fiber aperture.

The small diameter and flexibility of the fibers minimize 
tissue response, preserving local connectivity and circuit 
dynamics. Histology and immunohistochemistry from implants 
into zebra finch basal ganglia (depth 2.9~mm) show the 
splaying of the fibers and the presence of NeuN-stained 
cells in close proximity to the fiber tips.

By modeling the optical properties of the fibers and 
tissue, we simulate the interface properties of a bundle 
of fibers. Overlap in the sensitivity between nearby 
fibers allows application of blind source separation to 
extract individual neural traces. We describe a nonnegative 
independent component analysis algorithm especially suited 
to the interface.

Finally, experimental data from implants in transgenic 
mice yield proof of principle recordings during both 
cortical spreading depolarization and forepaw 
stimulation.

Collectively, the data presented here paint a compelling 
picture of splaying microfibers as a deep brain interface 
capable of sampling or perturbing neural activity at 
hundreds or thousands of points throughout a 3D volume of 
tissue while eliciting less response than existing optical 
implants.

\end{abstractpage}
\cleardoublepage

% Now you can include a preface. Again, use something like
% \newpage\section*{Preface} followed by your text

% Table of contents comes after preface
\tableofcontents
\cleardoublepage

% If you do not have tables, comment out the following lines
%\newpage
%\listoftables
%\cleardoublepage

% If you have figures, uncomment the following line
\newpage
\listoffigures
\cleardoublepage

% List of Abbrevs is NOT optional (Martha Wellman likes all abbrevs listed)
\chapter*{List of Abbreviations}
\begin{center}
  \begin{tabular}{lll}
    \hspace*{2em} & \hspace*{1in} & \hspace*{4.5in} \\
    ACSF & \dotfill & Artificial Cerebrospinal Fluid \\
    AST  & \dotfill & Abstract Syntax Tree \\
    AUC  & \dotfill & Area Under the Curve \\
    BMI  & \dotfill & Brain Machine Interface \\
    BSA  & \dotfill & Bovine Serum Albumin \\
    CDF  & \dotfill & Cumulative Density Function \\
    CMOS & \dotfill & Complementary Metal-Oxide-Semiconductor \\
    CNR  & \dotfill & Contrast-to-Noise Ratio \\
    CSD  & \dotfill & Cortical Spreading Depolarization \\
    CSF  & \dotfill & Cerebrospinal Fluid \\
    CSV  & \dotfill & Comma Separated Values \\
    DMD  & \dotfill & Digital Micromirror Device \\
    DPH  & \dotfill & Days Post Hatch \\
    FFT  & \dotfill & Fast Fourier Transform \\
    FPS  & \dotfill & Frames Per Second \\
    FITC & \dotfill & Fluorescein Isothiocyanate \\
    GECI & \dotfill & Genetically Encoded Calcium Indicator \\
    GFP  & \dotfill & Green Fluorescent Protein \\
    GRIN & \dotfill & Gradient-Index \\
    IACUC& \dotfill & Institutional Animal Care and Use Committee \\
    ICA  & \dotfill & Independent Component Analysis \\
    IV   & \dotfill & Intravenous \\
    LED  & \dotfill & Light Emitting Diode \\
    NA   & \dotfill & Numerical Aperture \\
    PBS  & \dotfill & Phosphate Buffered Saline \\
    PCA  & \dotfill & Principal Component Analysis \\
    ROC  & \dotfill & Receiver Operating Characteristic \\
    ROI  & \dotfill & Region of Interest \\
    SIFT & \dotfill & Scale-Invariant Feature Transform \\
    STFT & \dotfill & Short-Time Fourier Transform \\
    SNR  & \dotfill & Signal-to-Noise Ratio \\
    TTL  & \dotfill & Transistor-Transistor Logic \\
    USB  & \dotfill & Universal Serial Bus \\
  \end{tabular}
\end{center}
\cleardoublepage

% END OF THE PRELIMINARY PAGES

\newpage
\endofprelim
