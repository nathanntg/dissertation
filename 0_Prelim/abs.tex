% ABSTRACT

Optical interrogation and manipulation of neural dynamics 
is a cornerstone of systems neuroscience. Genetic targeting 
enable delivering fluorescent indicators and opsins to 
specific neural subpopulations. Optic probes can 
fluorescently sense and convey calcium, voltage, and 
neurotransmitter dynamics. This optical toolkit enables 
recording and perturbing cellular-resolution activity in 
thousands of neurons across a field of view.

Yet these techniques are limited by the light scattering 
properties of tissues. The cutting edge of microscopy, 
three-photon imaging, can record from intact tissues at 
depths up to 1 mm, but requires head-fixed experimental 
paradigms. To access deeper layers and non-cortical 
structures, researchers rely on optical implants, such as 
GRIN lenses or prisms, or the removal of superficial 
tissue. 

In this thesis, we introduce a novel implant for 
interfacing with deep brain regions constructed from 
bundles of hundreds or thousands of dissociated, small 
diameter ($<$8~\si{\micro\meter}) optical fibers. During 
insertion into the tissue, the fibers move independently, 
splaying through the target region. Each fiber achieves 
near total internal reflection, acting as a bidirectional 
optical interface with a small region of tissue near the 
fiber aperture.

The small diameter and flexibility of the fibers minimize 
tissue response, preserving local connectivity and circuit 
dynamics. Histology and immunohistochemistry from implants 
into zebra finch basal ganglia (depth 2.9~mm) show the 
splaying of the fibers and the presence of NeuN-stained 
cells in close proximity to the fiber tips.

By modeling the optical properties of the fibers and 
tissue, we simulate the interface properties of a bundle 
of fibers. Overlap in the sensitivity between nearby 
fibers allows application of blind source separation to 
extract individual neural traces. We describe a nonnegative 
independent component analysis algorithm especially suited 
to the interface.

Finally, experimental data from implants in transgenic 
mice yield proof of principle recordings during both 
cortical spreading depolarization and forepaw 
stimulation.

Collectively, the data presented here paint a compelling 
picture of splaying microfibers as a deep brain interface 
capable of sampling or perturbing neural activity at 
hundreds or thousands of points throughout a 3D volume of 
tissue while eliciting less response than existing optical 
implants.
