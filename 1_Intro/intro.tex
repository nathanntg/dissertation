\chapter{Introduction}
\label{chapter:intro}

\thispagestyle{myheadings}

Optical techniques have become a powerful tool in 
the process of observing and manipulating 
neural behavior at the systems level when combined 
with genetic probes that enable reading and writing 
neural activity at the action potential level 
\cite{Emiliani:2015jl}, as they provide a stable 
interface for interacting with cells over a large 
field of view. Novel genetic probes, such as voltage 
indicators, are only broadening the questions that can 
be answered via optical techniques by achieving 
faster time resolution and revealing subthreshold 
activity \cite{Gong:2015is}.

Despite these strengths, optical techniques are severely 
constrained by light scattering. In response to the 
limitations of scattering, a number of innovative approaches 
have been developed, with varying tradeoffs in terms of quality 
and feasibility. Multi-photon microscopy techniques enable 
imaging up to 1~mm below the surface, but require animals to 
be head-fixed and hence limit behavior paradigms 
\cite{Horton:2013gxa}. Attempts to adapt multi-photon microscopy 
to freely behaving animals have achieved neither the 
stability nor the ease of implementation desired 
\cite{Helmchen:2001tw,Flusberg:2005tq}.

Alternatively, more superficial brain regions can either 
be removed \cite{Dombeck:2010jr} or circumvented 
by implanting an optical probe, such as a GRIN lens 
\cite{Barretto:2009hk}, a microprism \cite{Andermann:2013kc}
or a communications-grade optical fiber for fiber 
photometry \cite{Guo:2015gu}. These techniques can be 
effective \cite{Betley:2015cn}, but damage superficial 
tissue and limit imaging to a planar cross-section 
of the desired brain region. 

In this thesis, we propose using bundles of splaying 
microfibers as a new optical interface, which has the 
potential to provide a high channel count, minimally 
invasive, stable, bidirectional optical interface for 
deep brain regions. The thesis is organized as follows: 
first, in \cref{chapter:intro}, we briefly review 
existing optical methods and associated tradeoffs, 
framing the role of the splaying optical microfibers in
there larger methodological landscape. Second, in 
\cref{chapter:histology}, we describe the implant in 
detail and histologically evaluate the implant, 
assessing both the splaying properties of the bundles
and the presence of neurons in close proximity to
the fibers. Next, we describe optical models of both 
the individual fibers and the interface properties of 
the bundles in \cref{chapter:modeling}. In 
\cref{chapter:recording}, we describe initial in vivo
applications of the interface and show fluorescence 
data recorded through the fibers, identifying. Finally, 
in \cref{chapter:processing}, we describe how peripheral 
developments (both hardware and software) that will enable 
application of the technique in longitudinal experimental 
work.

\section{Background}

\subsection{Optical techniques}

Thanks to advantages inherent to optical techniques for 
recording and manipulating neural activity, these techniques 
have become indispensable to advancing systems neuroscience. 
Specifically, optical techniques allow sensing and perturbing 
neural activity at a cellular spatial resolution and, due
to new indicators, at a sub action potential temporal 
resolution. In addition, optical techniques can be deployed 
in long-term experiments to track cellular dynamics over time 
in awake behaving animals, allowing directly probing the 
circuits relevant to complex behaviors.

\paragraph{Probes and indicators.} Crucial to the success of 
optical techniques are the development of genetically encoded 
probes. These proteins enable optically interfacing with 
existing cells. 

One of the earliest and most pervasive probes is GCaMP, a
combination of green fluorescent protein (GFP) isolated in 
jellyfish, and a calcium binding chain. When calcium 
binds to the protein, conformational changes in the chain 
alter the efficiency of the fluorescent protein and, 
as a result, increase its fluorescence
\cite{Nakai:2001fy,Barnett:2017kn}. By illuminating cells expressing such a
probe with the excitation wavelength and measuring the 
fluoresced emissions, it is possible to sense changes in
intracellular calcium and, hence, the calcium mediated 
depolarization associated with action potentials. Subsequent 
iterations on the protein have increased its efficiency, 
speed and brightness \cite{Chen:2013fc}.

The capacity to measure intracellular calcium enables 
inquiries into a range of signaling and encoding questions, 
but also is inherently limited to a suprathreshold cellular 
dynamics. New probes are being developed to sense other 
dimensions of cellular activity, including pH indicators, 
neurotransmitter indicators and voltage indicators 
\cite{Lin:2016id}. Voltage indicators have tremendous 
potential to allow visualizing both subthreshold and 
suprathreshold cellular dynamics at high temporal resolutions 
\cite{Han:2013iz,StPierre:2014db,Gong:2015is}. But given 
the highly localized voltage differentials (only present 
at the cell membrane) and the much shorter time course of 
voltage fluctuations, these probes must be more precisely
localized and much brighter to enable high frame rate 
acquisition and detection.

In addition to sensing other cellular dynamics, new fluorophores  
are being developed and deployed that leverage the inherent 
potential to multiplex signals across wavelengths. For example,
calcium and voltage indicators are being developed that use 
red or near-infrared fluorophores 
\cite{Tischbirek:2015fi,Dana:2016hx}. New fluorophores offer 
the potential to multiplex signals across non-overlapping 
wavelengths: this can enable sensing two properties 
simultaneously (e.g., calcium and voltage), or can enable sensing
from two subpopulations simultaneously, or can enable both 
sensing and modulating activity simultaneously. In addition to 
the ability to multiplex signals across wavelengths, new 
fluorophores can extend optical access to deeper brain regions by 
moving to wavelengths with lower absorption 
\cite{Tischbirek:2015fi,Dana:2016hx}.

Beyond sensing, novel genetic probes enable manipulating 
neural activity. Starting with Channelrhodopsin-1 
\cite{Nagel:2002cw}, which encodes a light-gated photon 
channel derived from the phototaxis mechanism in green 
algae, such probes have enabled modulating the excitability 
and membrane potential of neurons through controlled 
exposure to specific excitatory wavelengths 
\cite{Boyden:2005cd,Deisseroth:2006dd,Yizhar:2011jv}.

\paragraph{Genetic targeting.} All of these fluorescent probes 
and light gated channels are genetically encoded, which 
allows for precise targeting and expression of the protein 
in specific neural subpopulations of living animals. 
Transgenic animals are available with expression in cell
types of interest, while new viral vectors enable targeting 
expression in neurons based on connectivity \cite{Tervo:2016go}.
New techniques, such as relying on immediate early genes, 
allow targeting expression to those neurons involved 
in specific circuits and memories \cite{Liu:2012jv}. All 
of these advances in genetic targeting enable precisely
interfacing with a narrowly defined neural population, 
allowing researchers to interrogate the dynamics relevant 
to a specific behavior or circuit.

\paragraph{Field of view.} Optical interrogation and manipulation 
of neural circuits offers a number of additional advantages 
beyond the probe and targeting flexibility. One of the big 
advantages is the ability to interface with neurons over a 
large field of view, providing simultaneous access to thousands 
of neurons \cite{Mohammed:2016fq}. New iterations have scaled 
this up to hemisphere or skull sized imaging windows, exposing 
millions of neurons \cite{Kim:2016hh}. Through the large field 
of view, optical techniques provide the ability to track 
information encoding or behavioral modulation throughout a 
region or across regions, providing a more holistic vantage of
the neural activity. Of course, with this breadth of data, new 
analysis challenges emerge, as it becomes more difficult to 
identify the salient activity.

\paragraph{Longitudinal access.} Optical access not only 
achieves a large spatial field of view, it also enables long
term tracking of individual neurons. Cells can be tracked 
across days to understand how neural encoding and activity 
changes with time. For example, in recording calcium activity 
in the song bird premotor area exposed instability in 
neural encoding \cite{Liberti:2016bc} that was not visible in 
previous electrophysiology experiments, where single cells 
can often only be tracked on the scale of hours 
\cite{Hahnloser:2002hj}.

\paragraph{Awake behaving experiments.} Although inherent in 
long term access, it is important to emphasize the optically 
interfacing with the brain does not preclude awake behaving 
experiments. Many interface technologies, discussed in the 
next section, allow recording animals in freely behaving 
paradigms. In the song bird example just mentioned, this 
ability to record neural activity in freely behaving animals 
is crucial, as the birds will rarely sing if constrained 
\cite{LibertiIII:2017df}.

\subsection{Interface technology}

The last section identified 
a range of genetically encoded probes that enable optical 
access to neural activity. A complimentary area of 
technology development has focused on the hardware and 
techniques for recording or stimulating the described 
probes. There are a number of techniques used to precisely 
interface with the neurons in question, each of which 
achieves different tradeoffs in terms of sensitivity, 
signal localization and experimental constraints. We 
highlight three of the most widely used technologies 
below.

\paragraph{Multi-photon microscopy.} Multi-photon microscopy
provides powerful and precise access to fluorescent indicators,
achieving greater precision in the depth axis through the 
two photo absorption and improving penetration by exciting the 
tissue with higher wavelength tissue \cite{Xu:1996ul}. In addition,
the two-photon absorption decreases the amount of excitation light 
hitting the tissue, reducing the chance of \cite{Denk:1990ws}.
Three-photon microscopy further builds on these principles, 
achieving greater penetration, allowing imaging indicators 
at depths exceeding 1~mm \cite{Horton:2013gxa,Wang:2017jp}.

Despite these strengths, multi-photon microscopy faces key 
constraints. Given the cost associated with constructing and 
maintaining a two-photon microscope, there are substantial fixed 
costs and few labs have the capacity to run simultaneous 
experiments. In addition, the types of experiments are constrained,
often relying on recording in animals that are head fixed. 
Head-mounted two photon microscopes show 
potential \cite{Helmchen:2001tw,Flusberg:2005tq}, but these 
have not achieved the reliability and stability necessary for 
widespread use.

\paragraph{Single photon microscopy and miniature microscopes.}
Singe photon microscopy requires more excitation---risking more
photobleaching---and does not have the depth penetration or 
resolution of multi-photon microscopy, but it carries a number 
of important advantages. Most notably, single photon microscopy 
can readily be scaled to small applications, allowing recording 
from freely behaving animals through miniature head-mounted 
microscopes \cite{Ghosh:2011ee,Cai:2016hm,LibertiIII:2017df}.
These setups often come with a lower price tag and enable 
recording during less constrained behaviors.

Variations on the standard miniature microscopes are being 
developed that incorporate some of the advantages of multi-photon
microscopy, such as new head-mounted light-field microscopes 
capable of high speed volumetric imaging \cite{Skocek:2018hs}.

\paragraph{Fiber photometry.} % TODO: continue here

\cite{Adelsberger:2005dy,Cui:2013dq,Adelsberger:2014jd,Gunaydin:2014dh}

multi-site

\cite{Guo:2015gu}

\paragraph{Waveguides and new implant.} 

optoelectronic probes with embedded 
waveguides or with on-device, implantable light sources and 
detectors \cite{Warden:2014bx,Wu:2015gk,Segev:2017en}


%		Miniature microscopes
%		Multiphoton microscopy
%		Fiber photometry

\subsection{Accessing deep brain regions}
%		Challenge: scattering
%		Muliphoton microscopy
%		Implantable optics (GRIN lenses, prisms \cite{Andermann:2013kc}) - connectivity, scarring
%		Tissue removal

\section{Proposed Solution}
%		Bidirectional optical access
%		High channel count
%		Longitudinal access
%		Awake behaving access

% TODO: review onward

In order to translate the benefits inherent in superficial 
imaging techniques, such as long-term stability, high-channel 
counts and minimal tissue damage, to deep brain regions, we 
propose the use of arrays of dissociated optical microfibers 
that can be implanted in brain tissue. The principle of these 
optical fibers is based on self-splaying carbon fiber microthread 
arrays developed for electrophysiology 
\cite{Guitchounts:2013bs,Markowitz:2015ko}. The optical fiber 
bundles contain hundreds or thousands of multi-mode 
optical microfibers as small as 6.8~\si{\micro\meter} in diameter, 
displacing significantly less brain tissue and minimizing 
contact between tissue and non-biological surfaces as 
compared with existing optical fiber implants which are
often 125~\si{\micro\meter} wide. As each fiber travels independently 
and finds a path of least resistance, the fibers 
separate and splay, expanding the potential recording area 
while minimizing tissue damage. Each fiber maintains near total 
internal reflection of light,
which can be used for fluorescence recording of neural activity or optical stimulation of optogenetic probes.

We have successfully used two techniques for constructing the optical microfiber bundles. Initial prototypes were
constructed from unraveled quartz roving pulled together in a glass pipette, which was then cut and polished on 
a rotary grinder to act as an imaging surface. Light confinement in this configuration is achieved through the 
mismatch in the index of refraction between the fibers and surrounding tissue. The second approach relies on sourcing 
commercial leeched fiber bundles produced as flexible endoscopes, which are composed of 8~\si{\micro\meter} glass microfibers, 
where each fiber has a low index of refraction cladding for improved light confinement (figure~\ref{fig:i-confinement}).
The numerical aperture of such fibers is 0.4, providing a half-angle of acceptance of 17.2$^{\circ}$ in gray matter.
Over a length of 40~\si{\centi\meter}, the fibers attenuate incident light from a 473nm laster by 13.11dB, due to both coupling the 
unfocused light into the microfiber and propagation loss. The NA and attenuation efficiency suggest the bundles will
be effective for both optical stimulation and recording. Through 
the splaying and, potentially, through staggered cutting of the individual microfibers, the implanted
fibers will interface with cells throughout a target brain region, providing a high-channel count interface
allowing recording from many neurons involved in a system behavior.

Outside of the brain, the fibers converge to an imaging surface where the fibers form a close-packed array that
is polished for interfacing with standard microscope objectives. Recordings of fluorescent beads diffusing around the
splaying fibers reveal minimal cross talk and a high signal to noise ratio.

% \begin{itemize}
% 	\item Discuss bundles
% 	\item Preliminary histology
% 	\item Image of bundle \& beads
% 	\item Suggestion of full duplex recording
% \end{itemize}

\paragraph{Recording and stimulation interface.}

% Interface
%   Rotary Scope
%   Software

To realize the splaying microfiber technique in experimental conditions, we will develop
a rotary single-photon microscope that enables recording from freely behaving animals with
minimal constraints on motion. The bundle of optical microfibers is lightweight, flexible 
(with a bending radius of less than 1~\si{\centi\meter} and sufficiently
long (up to 80~\si{\centi\meter}), allowing the optical equipment to
be located away from the animal. As a result, a high-quality and a high-sensitivity 
CMOS sensor can be used to record fluorescent emissions from the fiber. This sensor and the 
accompanying optics will 
be placed on a commutator (figure~\ref{fig:a2-schematic}) which, in conjunction with the length 
of the fiber, will allow
the animal to move and rotate within its housing carrying only the weight of the fiber bundle. 
Both the commutator and the microscope will be adaptations of existing
head-mounted fluorescent techniques developed and used within the lab.

The CMOS sensor will be connected to an existing suite of software developed in the lab that 
allows automating recording. In the case of zebra finches, recording is triggered by the 
bird's song. Video from the sensor will be immediately digitized and written to disk, and 
can be analyzed in near-realtime, enabling behavioral- or cellular-contingent feedback. In 
order to best work with the spatially incoherent data that will come from the splaying microfibers,
the software may need to be adapted to handle non-contiguous regions of interest and live
source separation based on a precalculated mixing matrix.

The rotary microscope interface and accompanying software will facilitate running experiments
using the microfiber bundles by providing a robust recording mechanism and the ability to conduct
feedback-driven experiments that elicit complex behaviors during recording. For example, by
providing aversive feedback to zebra finches, we will observe how the basal ganglia (area X)
is involved in the learning of a modified (pitch- or time-shifted) song \cite{Tumer:2007bi}.
